\begin{recomendations}

    A partir de los resultados obtenidos y las limitaciones identificadas durante el desarrollo de esta investigación, se proponen las siguientes recomendaciones para trabajos futuros:

    \begin{enumerate}
    \item \textbf{Implementación nativa de SST e ISST en Python:} Se recomienda desarrollar una versión completa de los algoritmos de SST e ISST directamente en Python. Esto permitiría aprovechar capacidades avanzadas de procesamiento paralelo, tanto a nivel de CPU como de GPU, con el fin de reducir de manera significativa los tiempos de ejecución observados y facilitar su integración en entornos clínicos o de investigación que requieren procesamiento en tiempo real.

    \item \textbf{Exploración de otros métodos numéricos:} Se sugiere investigar y evaluar nuevas técnicas numéricas para la mejora de imágenes médicas mediante la metodología experimental establecida en este trabajo. Este esfuerzo podría conducir a la identificación de transformadas o algoritmos alternativos que proporcionen mejoras adicionales en la calidad de imagen o en la eficiencia computacional, sin incurrir en la complejidad y requisitos de datos de los enfoques basados en inteligencia artificial.

    \item \textbf{Desarrollo de métodos híbridos basados en IA y SST:} Se propone crear esquemas híbridos que combinen la transformada Synchrosqueezed Curvelet con modelos de inteligencia artificial. En particular, se recomienda entrenar mediante aprendizaje por refuerzo un modelo capaz de predecir la máscara óptima que se aplicaría sobre la energía obtenida tras la SST de una imagen determinada. Este enfoque presenta ventajas potenciales como la eliminación de la necesidad de datos etiquetados para el entrenamiento y una baja exigencia computacional, debido al reducido tamaño de las máscaras utilizadas. Aunque esta propuesta aún se considera hipotética, representa una línea de investigación prometedora que puede mejorar la adaptabilidad y el rendimiento del método desarrollado.
    \end{enumerate}
\end{recomendations}
