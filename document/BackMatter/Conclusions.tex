\begin{conclusions}
    Este trabajo desarrolló un método numérico basado en SST para la mejora de imágenes de CT. Las principales conclusiones obtenidas son:

    \begin{enumerate}
        \item \textbf{Implementación de SST-2DCT e ISST:} Se logró una implementación funcional de SST e ISST en Python 3.13 mediante \textit{oct2py}, permitiendo procesar imágenes NIfTI en entornos de cómputo general.

        \item \textbf{Diseño de experimentos:} Se evaluaron diferentes estrategias de modificación de la matriz de energía SST, lo que destaca la \textbf{Máscara Gaussiana ($\mu = N/2$, $\delta = 1$)} como la técnica con mejor balance entre realce de detalles y calidad de imagen.

        \item \textbf{Evaluación cuantitativa:} Las métricas aplicadas confirmaron que la técnica seleccionada mejora el contraste y la definición sin distorsiones ni artefactos significativos.

        \item \textbf{Comparación con otros métodos:} La propuesta superó a técnicas clásicas en preservación de detalles, aunque aún es inferior en desempeño global frente a métodos basados en inteligencia artificial entrenada con grandes volúmenes de datos.

        \item \textbf{Viabilidad computacional:} La solución es aplicable en entornos con recursos limitados, aunque los \textbf{tiempos de ejecución entre 2 y 7 minutos por imagen} representan una restricción para usos en tiempo real.
    \end{enumerate}

    \bigskip

    \textbf{Conclusión general:} El método basado en SST-2DCT constituye una alternativa efectiva y reproducible para mejorar el contraste en imágenes CT de cráneo sin recurrir a agentes de contraste ni a modelos de inteligencia artificial complejos. La técnica aporta un marco numérico que puede complementar enfoques basados en aprendizaje profundo, ofreciendo además transparencia e interpretabilidad. No obstante, persisten retos como la reducción de tiempos de ejecución y la automatización de parámetros, que deben abordarse en futuros desarrollos para facilitar su adopción clínica.
\end{conclusions}
