\begin{opinion}
    El uso de medios de contraste radiológicos en tomografía computarizada (TC) cerebral conlleva riesgos iatrogénicos significativos, como nefrotoxicidad, reacciones alérgicas e interferencias metabólicas, particularmente en pacientes con insuficiencia renal o comorbilidades. Estos efectos adversos justifican el empleo de métodos de mejoramiento digital del contraste o técnicas de inteligencia artificial, que permiten realzar estructuras anatómicas sin necesidad de aumentar la dosis de contraste.

    En particular, la transformada curvelet destaca en el procesamiento de imágenes de TC cerebral por su capacidad para representar eficientemente características morfológicas complejas. Su sensibilidad direccional multiescala permite capturar bordes curvilíneos y estructuras anatómicas finas, mejorando la detección de patologías. Además, su diseño anisotrópico separa eficazmente señal y ruido, preservando detalles clínicamente relevantes mientras reduce artefactos, optimizando tareas como segmentación, reconstrucción comprimida y mejora de contraste.

    Por su parte, la transformada synchrosqueezed (SST) ofrece alta resolución tiempo-frecuencia, descomponiendo señales no estacionarias con precisión. Localiza y separa componentes espectrales superpuestos, identificando patrones sutiles asociados a patologías. Su capacidad para reducir ruido manteniendo bordes y texturas la hace ideal para realzar estructuras cerebrales sin artefactos, siendo útil en diagnóstico asistido y procesamiento avanzado.

    El objetivo de la tesis fue desarrollar un método numérico basado en SST aplicada a curvelets (2DCT) para mejorar la calidad de imágenes de tomografía computarizada del cerebro, con especial énfasis en la visualización de tejidos blandos y lesiones pequeñas sin necesidad de agentes de contraste.

    La investigación realizada mostró que ventajas y desventajas del uso de este enfoque, que es una alternativa de menores prestaciones computacionales con respecto a las técnicas de inteligencia artificial, que no requiere un entrenamiento previo de modelos. Se recomienda continuar el trabajo en las áreas de mejora identificadas, tomando en cuenta el criterio cualitativo brindado por los especialistas.

    La tesis forma parte de las tareas de un proyecto nacional del CITMA para desarrollar algoritmos que aplican transformadas tiempo-frecuencia para analizar imágenes médicas, que incluye especialistas de alto nivel en diversas instituciones de salud cubanas y profesores e investigadores de nuestra facultad.

    Para esta tesis Pedro tuvo que estudiar las materias referidas, no incluidas en el currículo de la carrera, mostró disciplina, entrega y rigor. Además, demostró habilidades para el trabajo con la bibliografía y creatividad para proponer soluciones a problemas de implementación, entre otras competencias de programación en el lenguaje Python y sus diversos frameworks. En suma, considero que el estudiante logró cumplir el objetivo.

    \vspace{0.5cm}

    Por tanto, considero que a esta tesis del estudiante Pedro Pablo Álvarez Portelles debe otorgársele la máxima calificación (5 puntos, Excelente), y estoy seguro que en el futuro Pedro se desempeñará como un excelente profesional de la Computación.

    \vspace{1cm}

    \noindent
    Dr.C. Damian Valdés Santiago \\
    MSc. Dra. Lisbel Garzón Cutiño

    \vspace{0.5cm}

    \noindent
    13 de junio de 2025
\end{opinion}