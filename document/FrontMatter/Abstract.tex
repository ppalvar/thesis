\begin{resumen}
	La tomografía computarizada (CT) de cráneo es una herramienta esencial en el diagnóstico de patologías intracraneales, aunque su utilidad puede verse limitada por la presencia de imágenes con bajo contraste, especialmente en regiones de tejidos blandos y estructuras anatómicas sutiles. En este trabajo, se propone el uso de la transformada synchrosqueezed (SST), una técnica avanzada de procesamiento de señales, para mejorar la representación tiempo-frecuencia de la descomposición resultante de una transformada de curvelets 2D (2DCT). A diferencia de métodos tradicionales de mejora de imagen, SST permite una descomposición más precisa de componentes morfológicos y una reconstrucción adaptativa, preservando bordes y texturas críticas para el diagnóstico.  El estudio incluyó un análisis comparativo entre la SST y enfoques convencionales, evaluando métricas cuantitativas agrupadas en tres categorías: (1) Mejora, mediante el Índice de Mejora de Contraste (CII), la Relación de Nitidez Laplaciana (LSR) y el Índice de Similitud de Características (FSIM); (2) Distorsión, utilizando el PSNR (Relación Señal-Ruido Pico), SSIM (Índice de Similitud Estructural) y LPIPS (Percepción de Pérdida en Espacio de Características); y (3) Detección de Artefactos, mediante la Desviación Estándar del Ruido Residual (RNS) y el Índice BRISQUE (Evaluación de Calidad sin Referencia). Adicionalmente, se realizó una validación cualitativa por parte de especialistas.
\end{resumen}

\begin{abstract}
	Cranial computed tomography (CT) is an essential tool for diagnosing intracranial pathologies; however, its usefulness can be limited by low-contrast images, particularly in soft tissue regions and subtle anatomical structures. This work proposes the use of the synchrosqueezed transform (SST), an advanced signal processing technique, to enhance the time-frequency representation derived from a 2D curvelet transform (2DCT). Unlike traditional image enhancement methods, SST enables more precise decomposition of morphological components and adaptive reconstruction, preserving edges and textures critical for diagnosis. The study included a comparative analysis between SST and conventional approaches, evaluating quantitative metrics grouped into three categories: (1) Enhancement, using the Contrast Improvement Index (CII), Laplacian Sharpness Ratio (LSR), and Feature Similarity Index (FSIM); (2) Distortion, assessed via PSNR (Peak Signal-to-Noise Ratio), SSIM (Structural Similarity Index), and LPIPS (Learned Perceptual Image Patch Similarity); and (3) Artifact Detection, measured by Residual Noise Standard Deviation (RNS) and BRISQUE (Blind/Referenceless Image Spatial Quality Evaluator). Additionally, qualitative validation was performed by specialists.
\end{abstract}