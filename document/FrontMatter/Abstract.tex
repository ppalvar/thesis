\begin{resumen}
	La tomografía computarizada de cráneo es una herramienta esencial en el diagnóstico de patologías intracraneales, aunque su utilidad puede verse limitada por la presencia de imágenes con bajo contraste, especialmente en regiones de tejidos blandos y estructuras anatómicas sutiles. En este trabajo se propone el uso de la transformada \textit{synchrosqueezed}, una técnica avanzada de procesamiento de señales, para mejorar la representación tiempo-frecuencia de la descomposición resultante de una transformada \textit{curvelet} 2D. A diferencia de métodos tradicionales de mejora de imagen, SST permite una descomposición más precisa de componentes morfológicos y una reconstrucción adaptativa, lo que preserva bordes y texturas críticas para el diagnóstico.  El estudio incluyó un análisis estadístico, donde se calcularon métricas cuantitativas. Adicionalmente, se realizó una validación cualitativa por parte de una especialista.

	El método propuesto constituye una alternativa efectiva y accesible para el mejoramiento del contraste en imágenes de tomografía computarizada de cráneo, especialmente útil en situaciones donde no es posible el uso de agentes de contraste o tecnologías de inteligencia artificial avanzadas. Los resultados obtenidos evidencian que es posible realzar detalles anatómicos clínicamente relevantes, como lesiones pequeñas y estructuras de bajo contraste, sin introducir artefactos visuales significativos ni comprometer la integridad de la imagen original.
\end{resumen}

\begin{abstract}
	Cranial computed tomography is an essential tool in the diagnosis of intracranial pathologies, although its usefulness may be limited by the presence of low-contrast images, especially in soft tissue regions and subtle anatomical structures. This work proposes the use of the \textit{synchrosqueezed} transform, an advanced signal processing technique, to improve the time-frequency representation of the decomposition resulting from a 2D \textit{curvelet} transform. Unlike traditional image enhancement methods, SST allows for a more precise decomposition of morphological components and adaptive reconstruction, preserving edges and textures critical for diagnosis. The study included a statistical analysis, where quantitative metrics were calculated. Additionally, qualitative validation was performed by a specialist.

	The proposed method constitutes an effective and accessible alternative for contrast enhancement in cranial computed tomography images, particularly useful in situations where the use of contrast agents or advanced artificial intelligence technologies is not feasible. The obtained results demonstrate that it is possible to enhance clinically relevant anatomical details, such as small lesions and low-contrast structures, without introducing significant visual artifacts or compromising the integrity of the original image.
\end{abstract}