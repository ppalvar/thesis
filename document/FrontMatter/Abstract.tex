\begin{resumen}
	La tomografía computarizada (CT por sus siglas en inglés) de cráneo es una herramienta esencial en el diagnóstico de patologías intracraneales, aunque su utilidad puede verse limitada por la presencia de imágenes con bajo contraste, especialmente en regiones de tejidos blandos y estructuras anatómicas sutiles. En este trabajo se propone el uso de la transformada \textit{synchrosqueezed} (SST por sus siglas en inglés), una técnica avanzada de procesamiento de señales, para mejorar la representación tiempo-frecuencia de la descomposición resultante de una transformada de curvelets 2D (2DCT por sus siglas en inglés). A diferencia de métodos tradicionales de mejora de imagen, SST permite una descomposición más precisa de componentes morfológicos y una reconstrucción adaptativa, preservando bordes y texturas críticas para el diagnóstico.  El estudio incluyó un análisis estadístico, donde se calcularon métricas cuantitativas agrupadas en tres categorías: (1) Mejora, mediante el Índice de Mejora de Contraste (CII por sus siglas en inglés), la Relación de Nitidez Laplaciana (LSR por sus siglas en inglés) y el Índice de Similitud de Características (FSIM por sus siglas en inglés); (2) Distorsión, utilizando el PSNR (Relación Señal-Ruido Pico), Índice de Similitud Estructural (SSIM por sus siglas en inglés) y Percepción de Pérdida en Espacio de Características (LPIPS); y (3) Detección de Artefactos, mediante la Desviación Estándar del Ruido Residual (RNS por sus siglas en inglés) y el Índice de Evaluación de Calidad sin Referencia (BRISQUE por sus siglas en inglés). Adicionalmente, se realizó una validación cualitativa por parte de una especialista.
\end{resumen}

\begin{abstract}
	Cranial computed tomography (CT) is an essential tool in the diagnosis of intracranial pathologies, although its usefulness may be limited by the presence of low-contrast images, especially in regions of soft tissue and subtle anatomical structures. In this work, the use of the \textit{synchrosqueezed transform} (SST), an advanced signal processing technique, is proposed to improve the time-frequency representation of the decomposition resulting from a two-dimensional curvelet transform (2DCT). Unlike traditional image enhancement methods, SST allows for a more precise decomposition of morphological components and adaptive reconstruction, preserving edges and textures that are critical for diagnosis. The study included a statistical analysis, where quantitative metrics were calculated and grouped into three categories: (1) Enhancement, using the Contrast Improvement Index (CII), the Laplacian Sharpness Ratio (LSR), and the Feature Similarity Index (FSIM); (2) Distortion, using PSNR (Peak Signal-to-Noise Ratio), Structural Similarity Index (SSIM), and Learned Perceptual Image Patch Similarity (LPIPS); and (3) Artifact Detection, using the Residual Noise Standard Deviation (RNS) and the Blind/Referenceless Image Spatial Quality Evaluator (BRISQUE). Additionally, a qualitative validation was performed by a specialist.

\end{abstract}