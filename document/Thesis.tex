\documentclass[12pt,oneside]{uhthesis}
\usepackage{subfigure}
\usepackage[ruled,lined,linesnumbered,titlenumbered,algochapter,spanish,onelanguage]{algorithm2e}
\usepackage{amsmath}
\usepackage{amssymb}
\usepackage{amsbsy}
\usepackage{caption,booktabs}
\captionsetup{ justification = centering }
%\usepackage{mathpazo}
\usepackage{float}
\setlength{\marginparwidth}{2cm}
\usepackage{todonotes}
\usepackage{listings}
\usepackage{xcolor} 
\usepackage{multicol}
\usepackage{graphicx}

\usepackage{multirow} 
\usepackage{array}    
\usepackage{booktabs} 

\floatstyle{plaintop}
\restylefloat{table}
\addbibresource{Bibliography.bib}
% \setlength{\parskip}{\baselineskip}%
\renewcommand{\tablename}{Tabla}
\renewcommand{\listalgorithmcfname}{Índice de Algoritmos}
%\dontprintsemicolon
\SetAlgoNoEnd

\definecolor{codegreen}{rgb}{0,0.6,0}
\definecolor{codegray}{rgb}{0.5,0.5,0.5}
\definecolor{codepurple}{rgb}{0.58,0,0.82}
\definecolor{backcolour}{rgb}{0.95,0.95,0.92}

\lstdefinestyle{mystyle}{
    backgroundcolor=\color{backcolour},   
    commentstyle=\color{codegreen},
    keywordstyle=\color{purple},
    numberstyle=\tiny\color{codegray},
    stringstyle=\color{codepurple},
    basicstyle=\ttfamily\footnotesize,
    breakatwhitespace=false,         
    breaklines=true,                 
    captionpos=b,                    
    keepspaces=true,                 
    numbers=left,                    
    numbersep=5pt,                  
    showspaces=false,                
    showstringspaces=false,
    showtabs=false,                  
    tabsize=4
}

\lstset{style=mystyle}

\title{Mejoramiento del contraste de tomografía de cráneo con transformada \textit{synchrosqueezed}}
\author{\\\vspace{0.25cm}Pedro Pablo Álvarez Portelles}
\advisor{\\\vspace{0.25cm}Dr.C. Damian Valdés Santiago\\MSc. Dra. Lisbel Garzón Cutiño}
\degree{Licenciado en Ciencia de la Computación}
\faculty{Facultad de Matemática y Computación}
\date{10 de junio de 2025\\\vspace{0.25cm}\href{https://github.com/ppalvar/thesis}{github.com/ppalvar/thesis}}
\logo{Graphics/uhlogo}
\makenomenclature

\renewcommand{\vec}[1]{\boldsymbol{#1}}
\newcommand{\diff}[1]{\ensuremath{\mathrm{d}#1}}
\newcommand{\me}[1]{\mathrm{e}^{#1}}
\newcommand{\pf}{\mathfrak{p}}
\newcommand{\qf}{\mathfrak{q}}
\newcommand{\kt}{\mathtt{k}}
\newcommand{\mf}{\mathfrak{m}}
\newcommand{\hf}{\mathfrak{h}}
\newcommand{\fac}{\mathrm{fac}}
\newcommand{\maxx}[1]{\max\left\{ #1 \right\} }
\newcommand{\minn}[1]{\min\left\{ #1 \right\} }
\newcommand{\lldpcf}{1.25}
\newcommand{\nnorm}[1]{\left\lvert #1 \right\rvert }

\renewcommand{\lstlistoflistings}{} 
\renewcommand{\lstlistlistingname}{} 
\let\oldlstlistoflistings\lstlistoflistings
\renewcommand{\lstlistoflistings}{\relax}

\begin{document}

\frontmatter
\maketitle

\begin{dedication}
    Dedicado a mi madre, mi hermana, mi padre, mi familia y al amor de mi vida: los que han estado y estarán siempre conmigo.
\end{dedication}
\begin{acknowledgements}
    Expreso mi agradecimiento a la Universidad de La Habana y a su claustro de profesores, quienes con dedicación y entrega me han guiado hacia metas que en un principio parecían inalcanzables. Agradezco también a mis compañeros y amigos por su apoyo constante a lo largo de esta etapa. De manera especial, deseo reconocer la orientación y el acompañamiento de mi tutor, cuyo consejo ha sido fundamental para la realización de este trabajo.
\end{acknowledgements}
\begin{opinion}
    El uso de medios de contraste radiológicos en tomografía computarizada (TC) cerebral conlleva riesgos iatrogénicos significativos, como nefrotoxicidad, reacciones alérgicas e interferencias metabólicas, particularmente en pacientes con insuficiencia renal o comorbilidades. Estos efectos adversos justifican el empleo de métodos de mejoramiento digital del contraste o técnicas de inteligencia artificial, que permiten realzar estructuras anatómicas sin necesidad de aumentar la dosis de contraste.

    En particular, la transformada curvelet destaca en el procesamiento de imágenes de TC cerebral por su capacidad para representar eficientemente características morfológicas complejas. Su sensibilidad direccional multiescala permite capturar bordes curvilíneos y estructuras anatómicas finas, mejorando la detección de patologías. Además, su diseño anisotrópico separa eficazmente señal y ruido, preservando detalles clínicamente relevantes mientras reduce artefactos, optimizando tareas como segmentación, reconstrucción comprimida y mejora de contraste.

    Por su parte, la transformada synchrosqueezed (SST) ofrece alta resolución tiempo-frecuencia, descomponiendo señales no estacionarias con precisión. Localiza y separa componentes espectrales superpuestos, identificando patrones sutiles asociados a patologías. Su capacidad para reducir ruido manteniendo bordes y texturas la hace ideal para realzar estructuras cerebrales sin artefactos, siendo útil en diagnóstico asistido y procesamiento avanzado.

    El objetivo de la tesis fue desarrollar un método numérico basado en SST aplicada a curvelets (2DCT) para mejorar la calidad de imágenes de tomografía computarizada del cerebro, con especial énfasis en la visualización de tejidos blandos y lesiones pequeñas sin necesidad de agentes de contraste.

    La investigación realizada mostró que ventajas y desventajas del uso de este enfoque, que es una alternativa de menores prestaciones computacionales con respecto a las técnicas de inteligencia artificial, que no requiere un entrenamiento previo de modelos. Se recomienda continuar el trabajo en las áreas de mejora identificadas, tomando en cuenta el criterio cualitativo brindado por los especialistas.

    La tesis forma parte de las tareas de un proyecto nacional del CITMA para desarrollar algoritmos que aplican transformadas tiempo-frecuencia para analizar imágenes médicas, que incluye especialistas de alto nivel en diversas instituciones de salud cubanas y profesores e investigadores de nuestra facultad.

    Para esta tesis Pedro tuvo que estudiar las materias referidas, no incluidas en el currículo de la carrera, mostró disciplina, entrega y rigor. Además, demostró habilidades para el trabajo con la bibliografía y creatividad para proponer soluciones a problemas de implementación, entre otras competencias de programación en el lenguaje Python y sus diversos frameworks. En suma, considero que el estudiante logró cumplir el objetivo.

    \vspace{0.5cm}

    Por tanto, considero que a esta tesis del estudiante Pedro Pablo Álvarez Portelles debe otorgársele la máxima calificación (5 puntos, Excelente), y estoy seguro que en el futuro Pedro se desempeñará como un excelente profesional de la Computación.

    \vspace{1cm}

    \noindent
    Dr.C. Damian Valdés Santiago \\
    MSc. Dra. Lisbel Garzón Cutiño

    \vspace{0.5cm}

    \noindent
    13 de junio de 2025
\end{opinion}
\begin{resumen}
	La tomografía computarizada (CT) de cráneo es una herramienta esencial en el diagnóstico de patologías intracraneales, aunque su utilidad puede verse limitada por la presencia de imágenes con bajo contraste, especialmente en regiones de tejidos blandos y estructuras anatómicas sutiles. En este trabajo, se propone el uso de la transformada synchrosqueezed (SST), una técnica avanzada de procesamiento de señales, para mejorar la representación tiempo-frecuencia de la descomposición resultante de una transformada de curvelets 2D (2DCT). A diferencia de métodos tradicionales de mejora de imagen, SST permite una descomposición más precisa de componentes morfológicos y una reconstrucción adaptativa, preservando bordes y texturas críticas para el diagnóstico.  El estudio incluyó un análisis comparativo entre la SST y enfoques convencionales, evaluando métricas cuantitativas agrupadas en tres categorías: (1) Mejora, mediante el Índice de Mejora de Contraste (CII), la Relación de Nitidez Laplaciana (LSR) y el Índice de Similitud de Características (FSIM); (2) Distorsión, utilizando el PSNR (Relación Señal-Ruido Pico), SSIM (Índice de Similitud Estructural) y LPIPS (Percepción de Pérdida en Espacio de Características); y (3) Detección de Artefactos, mediante la Desviación Estándar del Ruido Residual (RNS) y el Índice BRISQUE (Evaluación de Calidad sin Referencia). Adicionalmente, se realizó una validación cualitativa por parte de especialistas.
\end{resumen}

\begin{abstract}
	Cranial computed tomography (CT) is an essential tool for diagnosing intracranial pathologies; however, its usefulness can be limited by low-contrast images, particularly in soft tissue regions and subtle anatomical structures. This work proposes the use of the synchrosqueezed transform (SST), an advanced signal processing technique, to enhance the time-frequency representation derived from a 2D curvelet transform (2DCT). Unlike traditional image enhancement methods, SST enables more precise decomposition of morphological components and adaptive reconstruction, preserving edges and textures critical for diagnosis. The study included a comparative analysis between SST and conventional approaches, evaluating quantitative metrics grouped into three categories: (1) Enhancement, using the Contrast Improvement Index (CII), Laplacian Sharpness Ratio (LSR), and Feature Similarity Index (FSIM); (2) Distortion, assessed via PSNR (Peak Signal-to-Noise Ratio), SSIM (Structural Similarity Index), and LPIPS (Learned Perceptual Image Patch Similarity); and (3) Artifact Detection, measured by Residual Noise Standard Deviation (RNS) and BRISQUE (Blind/Referenceless Image Spatial Quality Evaluator). Additionally, qualitative validation was performed by specialists.
\end{abstract}
\include{FrontMatter/Contents}

\mainmatter

\chapter*{Introducción}\label{chapter:introduction}
En la actualidad, las herramientas diagnósticas han alcanzado un notable desarrollo tecnológico, lo que permite una detección más precisa de diversas enfermedades \cite{semmlow2008biosignal}, sin embargo, muchos de los procedimientos utilizados son altamente invasivos para los pacientes lo que compromete su salud en muchos casos, y en ocasiones no es suficientemente efectivo para poder realizar el diagnóstico.

En este contexto, la tomografía computarizada (CT, por sus siglas en inglés) es una poderosa herramienta para la visualización indirecta de los tejidos y órganos internos de los pacientes, sin necesidad de una observación directa mediante acceso quirúrgico. Sin embargo, cuando se trata de observar tejidos blandos y lesiones pequeñas (e.g., tumores o hemorragias de varios milímetros en zonas del cerebro) suele ser necesario introducir al organismo un agente contrastante radiactivo como una solución de yodo \cite{InsideRadiologyICCM}.

Este método es efectivo en muchos casos, ya que proporciona una imagen con mejoras en el contraste, principalmente en zonas con múltiples vasos sanguíneos. Por otro lado, el método puede tener complicaciones para la salud de los pacientes con enfermedades tiroideas o, en casos extremos, provocar ataques cardíacos \cite{IodineTyroids,IodineHeathAttack}.

Dado que las tomografías, en muchos casos, son capaces de captar los detalles necesarios para realizar un diagnóstico, pero estos no son apreciables debido a problemas de contraste, ruido o emborronamiento de la imagen, resulta necesario la mejora digital de estas imágenes mediante métodos computacionales precisos, en particular, imágenes de CT del cerebro, que al ser un órgano gelatinoso donde las lesiones suelen ser pequeñas. 

En la literatura se reportan métodos de mejora de imágenes que utilizan redes neuronales en alguna de sus variantes \cite{ULTRA,DLR,LEARN++,EDCNN}. Este es el caso también de las imágenes de CT, en las cuales se logran resultados muy satisfactorios en ese campo. Los estudios más recientes (2022-2025) demuestran que las técnicas basadas en inteligencia artificial (IA, por sus siglas) mejoran significativamente la calidad de imagen y mantienen una alta precisión diagnóstica, incluso en estudios con dosis reducidas de radiación \cite{AISurveyOnImageQuality}.

Aunque los algoritmos de IA han demostrado potencial en el diagnóstico médico, su implementación efectiva requiere grandes volúmenes de datos anotados por especialistas, así como recursos computacionales avanzados. En contextos con limitaciones de infraestructura —como el sistema de salud cubano—, estas condiciones rara vez se cumplen, especialmente en el caso de bases de datos imagenológicas de CT.

Por ello, resulta crítico explorar métodos numéricos tradicionales y novedosos, los cuales, a pesar de ofrecer resultados inferiores en algunos escenarios , presentan ventajas clave: menor dependencia de datos, menor costo computacional y posibilidad de despliegue en dispositivos móviles.

En este contexto, uno de los métodos numéricos de procesamiento de imágenes más actuales es la transformada \emph{curvelet} (2DCT, por sus siglas en inglés), que ofrece mayor precisión para analizar imágenes con respecto a la transformada \emph{wavelet} tradicional \cite{Flandrin2018}, pues puede detectar bordes en ángulos y distintas escalas \cite{FastCurveletTransform}. Sin embargo, al considerar las escalas y ángulos como una generalización de tiempo-frecuencia usado en el análisis \emph{wavelet}, la resolución temporal obtenida es limitada y ofrece grandes áreas de incertidumbre que dificultan la creación de algoritmos que utilicen directamente la 2DCT.

La transformada de \emph{synchrosqueezed} (SST, por sus siglas en inglés) es un método de reasignación de frecuencias que permite refinar la representación obtenida mediante 2DCT \cite{SynchrosqueezedCurveletTransform}. En este método se obtiene una matriz de coeficientes (también referida como energía SST) que representa la energía de la imagen en cada escala y ángulo.

La hipótesis central de esta tesis es la siguiente: si se realiza una modificación de la energía SST, ¿se puede obtener una mejora en la calidad de la imagen al aplicar la función inversa (ISST) de la transformada?

El presente trabajo de tesis tiene como objetivo principal desarrollar un método numérico basado en SST aplicada a curvelets (2DCT) para mejorar la calidad de imágenes de tomografía computarizada del cerebro, con especial énfasis en la visualización de tejidos blandos y lesiones pequeñas sin necesidad de agentes de contraste.

Como objetivos específicos se plantean:

\begin{enumerate}
    \item Implementar la transformada SST-2DCT e ISST sobre imágenes de CT cerebral con \texttt{Python 3.13}.
    \item Diseñar y ejecutar experimentos enfocados a tener una mejor representación de las características deseadas de la imagen, mediante la modificación de sus matrices de energía.
    \item Evaluar cuantitativamente los resultados mediante métricas de mejora, a través de conjuntos de métricas enfocadas en medir la mejora, la distorsión y el ruido de los resultados.
    \item Comparar la reconstrucción inversa (ISST) optimizada con métodos tradicionales de mejoramiento de imágenes de CT y métodos más modernos.
    \item Analizar la viabilidad del método en entornos con recursos computacionales limitados y proponer recomendaciones de implementación para futuros trabajos.
\end{enumerate}

Con estos objetivos se busca: (1) optimizar la reconstrucción inversa (ISST) para preservar características clínicamente relevantes, (2) reducir artefactos y distorsiones en las imágenes procesadas, y (3) mejorar el contraste, nitidez y calidad general de la imagen para los estándares de un profesional del área.

Para simplificar los pasos de implementación, se utilizó la implementación de SST e ISST existentes en \texttt{SynLab} \cite{SynchrosqueezedCurveletTransform,SynchrosqueezedCurveletTransform_SynLab}, adaptado a \texttt{Python 3.13} mediante la librería \texttt{oct2py} \cite{oct2py}.

Para evaluar los resultados de los experimentos se utilizaron un conjunto de métricas agrupadas en tres categorías:

\begin{itemize}
    \item \textbf{Métricas de mejora}: miden el incremento en la calidad de la imagen.
    \item \textbf{Métricas de distorsión}: mide la distorsión de la imagen con respecto a la original, tanto en sus características semánticas como en otras como el ruido introducido.
    \item \textbf{Métricas de artefactos}: miden cuántos artefactos tiene la imagen reconstruida con respecto a la original.
\end{itemize}

Los resultados obtenidos podrían ofrecer una alternativa no invasiva para mejorar el diagnóstico por imágenes, particularmente en pacientes con contraindicaciones para el uso de contrastes yodados, lo que contribuye al avance de técnicas computacionales en radiología médica.

En el contexto actual de rápida adopción de la IA en el diagnóstico médico, este trabajo aporta dos contribuciones clave: primero, establece un marco numérico reproducible que podría servir como capa de preprocesamiento para modelos de aprendizaje profundo. Esto pudiera potenciar su eficiencia al reducir la complejidad del espacio de características. Segundo, provee un enfoque interpretable que mitiga el frecuente ``efecto caja negra'' de los sistemas basados puramente en IA.

Los resultados podrían influir en el desarrollo de sistemas híbridos IA-métodos numéricos, particularmente relevantes para entornos clínicos con limitaciones tecnológicas o donde se priorice la transparencia diagnóstica. Además, la metodología propuesta podría extenderse a otras modalidades de imagen médica más allá de la tomografía computarizada.

Esta tesis está organizada en tres capítulos. El primero introduce los conceptos fundamentales para la comprensión del trabajo, incluyendo bases teóricas de procesamiento de imágenes, la transformada \emph{curvelet} y la transformada \emph{synchrosqueezed}. Además, presenta una revisión del estado del arte, donde se analizan herramientas existentes y seleccionando aquellas relevantes para esta investigación.

El segundo capítulo presenta en detalle la implementación del método propuesto para la mejora de imágenes. En esta sección se describen los conjuntos de datos empleados, las métricas de evaluación seleccionadas y la justificación de la metodología adoptada. Asimismo, se expone el procedimiento seguido para la integración de las técnicas de procesamiento de imágenes, especificando los parámetros utilizados y la estructura general del modelo desarrollado.

El tercer capítulo expone los resultados estadísticos obtenidos a partir de la experimentación. Se presentan de manera sistemática los resultados derivados de las distintas configuraciones evaluadas. Además, los resultados fueron sometidos a la evaluación de una especialista en el área, con el fin de validar la calidad de las imágenes procesadas desde una perspectiva experta. Finalmente, se realiza un análisis crítico de los hallazgos, resaltando las ventajas y limitaciones observadas, y se ofrecen recomendaciones para futuras líneas de investigación.

Esta tesis forma parte del proyecto de investigación ``Wavelets, frames, técnicas espectrales, ecuaciones en derivadas parciales y aprendizaje automático científico en el análisis de imágenes'', asociado al Programa Nacional de Ciencias Básicas, Código PN223LH010-036, Ministerio de Ciencia, Tecnología y Medio Ambiente (CITMA), Cuba, 2024-2026.

\addcontentsline{toc}{chapter}{Introducción}

\chapter{Marco teórico}\label{chapter:state-of-the-art}

En este capítulo se presentan los fundamentos teóricos necesarios para el desarrollo de esta investigación. Se abordan los principios del procesamiento digital de imágenes, con especial énfasis en las características y particularidades de las imágenes obtenidas por tomografía computarizada (TC) de cráneo. Asimismo, se describen las técnicas convencionales y actuales de mejora de contraste en imágenes médicas, destacando sus ventajas y limitaciones. Finalmente, se introduce el marco conceptual de la transformada synchrosqueezed, que servirá de base para la propuesta metodológica de mejoramiento de contraste desarrollada en este trabajo.

\section{Representación de imágenes de CT}

En el contexto de esta tesis, una imagen digital se representa formalmente como una matriz $ M \in \mathbb{R}^{n \times m} $, donde cada elemento $ (i, j) $ corresponde a la intensidad o luminancia del píxel ubicado en la fila $ i $ y columna $ j $. En imágenes a color, la representación suele involucrar tres matrices independientes, cada una asociada a la intensidad de los canales rojo, verde y azul (RGB por sus siglas en inglés). Sin embargo, dado que las imágenes de CT son inherentemente monocromáticas, una sola matriz es suficiente para describir la distribución de intensidades, lo que simplifica su procesamiento y análisis.%todo: poner ref 

En el caso particular de las imágenes médicas obtenidas mediante tomografía computarizada de cráneo, cada valor de la matriz representa la atenuación de los rayos X en una región específica del tejido, cuantificada mediante unidades Hounsfield (HU, por sus siglas en inglés). Estas unidades permiten distinguir entre diferentes tipos de tejidos, como hueso, sustancia gris, sustancia blanca y líquido cefalorraquídeo, en función de sus propiedades de absorción.%todo: poner ref

Durante la adquisición de imágenes de tomografía computarizada (CT), se utiliza un equipo especializado compuesto por un escáner de gran tamaño con forma de anillo, denominado gantry. El paciente se recuesta sobre una mesa motorizada que se desplaza lentamente a través del gantry, mientras un tubo de rayos X y un conjunto de detectores electrónicos rotan alrededor de la cabeza del paciente. Este sistema emite haces de rayos X que atraviesan los tejidos y son atenuados en función de sus propiedades físicas; los detectores captan la radiación remanente y envían la información a una computadora central.%todo: poner ref

La computadora procesa los datos recolectados durante las múltiples rotaciones y posiciones del tubo de rayos X, aplicando algoritmos matemáticos avanzados para reconstruir imágenes transversales o cortes bidimensionales del cráneo. Estas imágenes pueden ser posteriormente apiladas para obtener representaciones tridimensionales detalladas, lo que facilita la identificación precisa de estructuras anatómicas y posibles patologías.%todo: poner ref

En el desarrollo de esta tesis, las imágenes médicas utilizadas se almacenan y procesan en el formato NIfTI (Neuroimaging Informatics Technology Initiative, con extensión de archivo \texttt{.nii}). Este formato fue diseñado específicamente para aplicaciones de neuroimagen, superando las limitaciones de formatos previos como Analyze y permitiendo el manejo eficiente de datos multidimensionales, como los obtenidos en estudios de tomografía computarizada (CT) de cráneo. NIfTI posibilita la representación de volúmenes completos, integrando en un solo archivo tanto la información de los datos de imagen como los metadatos relevantes para el análisis, como la orientación espacial y las dimensiones físicas de los vóxeles. Esta capacidad lo convierte en un estándar ampliamente adoptado en la investigación y el procesamiento avanzado de imágenes cerebrales, facilitando la interoperabilidad con herramientas especializadas de análisis y visualización.%todo: poner ref

NIfTI resuelve limitaciones importantes relacionadas con la representación de datos, como la incapacidad de manejar ciertos tipos de datos (por ejemplo, enteros sin signo de 16 bits) y la falta de información precisa sobre la orientación espacial de las imágenes. NIfTI permite almacenar tanto los datos de imagen como los metadatos relevantes en un único archivo o en archivos separados, facilitando la interoperabilidad entre diferentes plataformas y herramientas de análisis . Además, ofrece soporte nativo para imágenes multidimensionales, donde las tres primeras dimensiones corresponden a las coordenadas espaciales $ (x, y, z) $ y la cuarta puede ser utilizada para representar series temporales o parámetros adicionales, lo que resulta especialmente útil en estudios volumétricos y funcionales.

Entre las principales ventajas del formato NIfTI destaca su capacidad para asociar las coordenadas de la imagen con posiciones en el espacio real, mejorando la precisión en el análisis y la comparación entre distintos estudios. Asimismo, su adopción generalizada en la comunidad científica ha impulsado el desarrollo de herramientas especializadas para su visualización y procesamiento, facilitando la reproducibilidad y el intercambio de datos. En el contexto de esta tesis, se emplea un conjunto de datos proveniente de PhysioNet, denominado \emph{Computed Tomography Images for Intracranial Hemorrhage Detection and Segmentation} \cite{DatasetPhysionet,DatasetOriginalArticle}, el cual está disponible en formato NIfTI y proporciona imágenes de tomografía computarizada de cráneo adecuadas para la investigación y validación de técnicas de mejoramiento de contraste.

\section{Aspectos de procesamiento de imágenes}

\subsection{Parámetros fundamentales de la calidad de imagen}

Las imágenes médicas digitales presentan una serie de parámetros fundamentales que determinan su calidad y utilidad diagnóstica. Entre estos parámetros se encuentran el contraste, la resolución espacial, la nitidez y el nivel de ruido, los cuales influyen directamente en la percepción visual de las estructuras anatómicas y en la capacidad de los especialistas para identificar hallazgos relevantes. En esta subsección se describirán brevemente estos parámetros, así como su manifestación visual en las imágenes, proporcionando el marco necesario para comprender los procesos de mejoramiento y análisis aplicados en el procesamiento de imágenes médicas\cite{ImageProcessingBook}.

El contraste se refiere a la diferencia en la intensidad o brillo entre distintas áreas de una imagen, lo que permite distinguir claramente las estructuras anatómicas y detectar posibles anomalías. En el contexto de la tomografía computarizada (CT), el contraste es fundamental para resaltar tejidos con diferentes propiedades de absorción de rayos X, facilitando la identificación de órganos, vasos sanguíneos y lesiones. Para mejorar este contraste, en muchos casos se emplean medios de contraste, que son sustancias químicas administradas al paciente por vía oral, intravenosa o rectal, y que modifican temporalmente la forma en que los rayos X interactúan con los tejidos.

Estos agentes de contraste, como los compuestos yodados en CT, permiten que ciertas áreas del cuerpo absorban más o menos radiación, haciendo que aparezcan más claras u oscuras en la imagen final. De este modo, se mejora la diferenciación entre tejidos normales y patológicos, lo que incrementa la precisión diagnóstica. Aunque el uso de medios de contraste no es obligatorio en todos los estudios, su aplicación es crucial en la evaluación detallada de estructuras específicas y en la detección de lesiones que podrían pasar desapercibidas en imágenes sin contraste. Es importante señalar que, si bien estos medios son generalmente seguros, pueden presentar riesgos mínimos, por lo que su uso debe ser evaluado cuidadosamente por el especialista.

El ruido en las imágenes médicas se manifiesta como variaciones aleatorias e indeseadas en la intensidad de los píxeles, que no corresponden a las características reales de los tejidos o estructuras anatómicas. Este fenómeno puede tener su origen en múltiples factores, como las limitaciones físicas de los detectores, la radiación dispersa, el procesamiento digital y las condiciones de adquisición de la imagen. El ruido degrada la calidad visual, dificultando la identificación de detalles finos y reduciendo la relación señal-ruido, lo que puede comprometer la precisión diagnóstica. En las imágenes de tomografía computarizada, el tipo de ruido más común es el gaussiano, aunque pueden presentarse otras formas dependiendo del equipamiento y el protocolo utilizado. Para mitigar su impacto, se emplean diversas técnicas de reducción de ruido, como el filtrado espacial y métodos avanzados basados en inteligencia artificial, cuyo objetivo es preservar la información relevante sin eliminar detalles críticos para el diagnóstico.

La nitidez, por su parte, se refiere a la claridad con la que se representan los bordes y los detalles en una imagen. Una imagen nítida permite distinguir de manera precisa los límites entre distintas estructuras, lo que facilita la interpretación clínica y la detección de anomalías. La nitidez está directamente relacionada con la resolución espacial del sistema de adquisición y puede verse afectada por factores como el movimiento del paciente, el enfoque del detector y los algoritmos de reconstrucción empleados. Sin embargo, existe una relación inversa entre nitidez y ruido: al aumentar la nitidez, es posible que también se incremente el nivel de ruido, por lo que los sistemas de procesamiento de imágenes deben buscar un equilibrio adecuado entre ambos parámetros para garantizar la mejor calidad diagnóstica posible.

\subsection{Métodos clásicos de mejora de imágenes}

\subsubsection{Filtrado Gaussiano}

El filtrado Gaussiano\cite{GaussianFilter} es una técnica clásica de procesamiento de imágenes utilizada principalmente para la reducción de ruido aleatorio, como el ruido electrónico o de tipo Poisson, en imágenes médicas. Este método se basa en la convolución de la imagen original con una máscara o núcleo Gaussiano, que asigna mayor peso a los píxeles cercanos al centro de la ventana y menor peso a los más alejados. El resultado es un suavizado progresivo de la imagen, que atenúa las fluctuaciones de intensidad no deseadas manteniendo la continuidad de las estructuras anatómicas principales. El filtrado Gaussiano es especialmente útil como etapa de preprocesamiento antes de procedimientos como la segmentación o la visualización, ya que reduce el ruido sin eliminar completamente los bordes relevantes. Sin embargo, su principal limitación radica en la posible pérdida de detalles finos y la leve difuminación de los contornos, lo que requiere un ajuste cuidadoso del parámetro de desviación estándar de la función Gaussiana para equilibrar la reducción de ruido y la preservación de la información estructural.

\begin{figure}[H]
    \centering
    \includegraphics[width=0.95\textwidth]{Graphics/gaussian-filter.png}
    \caption{Comparación visual del efecto del filtrado Gaussiano en una imagen de tomografía computarizada de cráneo: (a) imagen original, (b) imagen tras la aplicación del filtro Gaussiano.}
    \label{fig:filter-gaussian}
\end{figure}

\subsubsection{Filtrado de Mediana}

El filtrado de mediana es otro método ampliamente utilizado para la mejora de imágenes médicas, particularmente eficaz en la eliminación de ruido impulsivo, como el conocido ``ruido sal y pimienta''. A diferencia de los filtros lineales, el filtrado de mediana es un proceso no lineal que reemplaza el valor de cada píxel por la mediana de los valores de intensidad de sus vecinos dentro de una ventana definida. Esta característica permite suprimir eficazmente los valores atípicos sin suavizar excesivamente los bordes ni perder detalles estructurales importantes. Por ello, el filtrado de mediana es especialmente valorado en aplicaciones donde la preservación de las estructuras finas, como vasos sanguíneos o límites tisulares, es prioritaria. No obstante, su desempeño puede verse afectado en presencia de grandes regiones de ruido o cuando se utilizan ventanas demasiado grandes, lo que podría conducir a la pérdida de información relevante\cite{ImageProcessingBook}.

\begin{figure}[H]
    \centering
    \includegraphics[width=0.95\textwidth]{Graphics/median-filter.png}
    \caption{Comparación visual del efecto del filtrado de mediana en una imagen de tomografía computarizada de cráneo: (a) imagen original, (b) imagen tras la aplicación del filtro de mediana.}
    \label{fig:filter-median}
\end{figure}

\subsubsection{Ecualización Adaptativa del Histograma (CLAHE)}

La ecualización adaptativa del histograma, conocida como CLAHE (Contrast Limited Adaptive Histogram Equalization \cite{CLAHE}), es una técnica avanzada de mejora de contraste que se utiliza ampliamente en el procesamiento de imágenes médicas. A diferencia de la ecualización global de histograma, que redistribuye las intensidades de toda la imagen de manera uniforme, CLAHE divide la imagen en pequeñas regiones o mosaicos y aplica la ecualización de histograma de forma local en cada uno de ellos. Este enfoque permite resaltar detalles en áreas específicas sin amplificar excesivamente el ruido ni crear artefactos indeseados, lo que resulta especialmente útil en imágenes con variaciones locales de contraste, como las obtenidas en tomografía computarizada.

CLAHE incorpora un parámetro de limitación de contraste (\emph{clip limit}) que controla el grado de realce permitido en cada mosaico, evitando la sobre-amplificación del ruido en regiones homogéneas. Además, el tamaño de los mosaicos (\emph{tileGridSize}) puede ajustarse para equilibrar el nivel de detalle local y el efecto global del contraste. Esta técnica ha demostrado ser eficaz para mejorar la visibilidad de estructuras sutiles en tejidos blandos o regiones con bajo contraste, facilitando el análisis y la interpretación clínica. Sin embargo, su aplicación debe realizarse con precaución, ya que un ajuste inadecuado de los parámetros puede introducir artefactos o modificar la apariencia de ciertas regiones relevantes para el diagnóstico.

\begin{figure}[H]
    \centering
    \includegraphics[width=0.95\textwidth]{Graphics/cahe.png}
    \caption{Comparación visual del efecto de la ecualización adaptativa del histograma en una imagen de tomografía computarizada de cráneo: (a) imagen original, (b) imagen tras la aplicación de CLAHE.}
    \label{fig:filter-clahe}
\end{figure}

\subsubsection{Transformación Homomórfica}

La transformación homomórfica \cite{HomomorphicFilter} es un método clásico orientado a la mejora simultánea del contraste y la nitidez en imágenes digitales. Su fundamento radica en el modelado de la imagen como el producto de dos componentes: la iluminación (de baja frecuencia) y la reflectancia (de alta frecuencia). Mediante una transformación logarítmica, este producto se convierte en una suma, permitiendo la aplicación de filtros en el dominio de la frecuencia para atenuar las variaciones lentas de iluminación y realzar los detalles finos asociados a los bordes y texturas.

En el contexto de imágenes médicas, la transformación homomórfica es especialmente útil para corregir problemas de iluminación no uniforme y para destacar estructuras anatómicas que podrían pasar desapercibidas en cortes oscuros o mal iluminados. Tras el procesamiento, se aplica la transformación exponencial inversa para reconstruir la imagen mejorada. Este método ofrece la ventaja de mejorar el contraste local y global de manera simultánea, incrementando la claridad de los bordes y la percepción de detalles relevantes para el diagnóstico. No obstante, la selección adecuada de los parámetros del filtro es crucial para evitar la introducción de artefactos y preservar la información diagnóstica esencial.

\begin{figure}[H]
    \centering
    \includegraphics[width=0.95\textwidth]{Graphics/homomorphic.png}
    \caption{Comparación visual del efecto de la transformación homomórfica en una imagen de tomografía computarizada de cráneo: (a) imagen original, (b) imagen tras la aplicación de la transformación homomórfica.}
    \label{fig:filter-homomorphic}
\end{figure}

\subsubsection{Detección de Bordes (Canny o Laplaciano)}

La detección de bordes es una técnica fundamental en el procesamiento de imágenes médicas, cuyo objetivo principal es resaltar los contornos y límites anatómicos presentes en la imagen \cite{CannyBorderDetection,LaplacianBorderDetection}. Los métodos clásicos, como el detector de Canny y el operador Laplaciano, permiten identificar transiciones abruptas de intensidad, que suelen corresponder a los bordes entre diferentes tejidos u órganos. La aplicación de estos algoritmos facilita la delimitación precisa de regiones de interés, como órganos, vasos sanguíneos o lesiones, lo que resulta esencial para tareas posteriores de segmentación, cuantificación y análisis morfológico.

El detector de Canny es especialmente valorado por su capacidad para localizar bordes de forma robusta y continua, minimizando la detección de falsos positivos gracias a su enfoque multietapa que incluye suavizado, cálculo de gradientes, supresión de no-máximos y umbralización con histéresis. Por otro lado, el operador Laplaciano, basado en la segunda derivada de la intensidad, destaca los puntos de cambio rápido en la imagen, aunque es más sensible al ruido y suele emplearse en combinación con técnicas de suavizado previo. En el contexto de la tomografía computarizada, la detección de bordes contribuye significativamente a la mejora de la visualización de estructuras anatómicas y a la precisión de los procesos de diagnóstico asistido por computadora, permitiendo una interpretación más clara y objetiva de las imágenes.

\begin{figure}[H]
    \centering
    \includegraphics[width=0.95\textwidth]{Graphics/canny.png}
    \caption{Comparación visual del efecto de la detección de bordes mediante el filtro de Canny en una imagen de tomografía computarizada de cráneo: (a) imagen original, (b) imagen tras la aplicación del detector de bordes Canny.}
    \label{fig:filter-canny}
\end{figure}

\subsection{Métodos estado del arte}

\subsubsection{EDCNN}

Uno de los avances recientes en la mejora de imágenes de tomografía computarizada (CT) de baja dosis es el modelo EDCNN (Edge enhancement-based Densely Connected Network \cite{EDCNN}), una red neuronal convolucional diseñada específicamente para la reducción de ruido manteniendo la integridad de los detalles anatómicos. EDCNN introduce un módulo de mejora de bordes que utiliza operadores Sobel entrenables para extraer y realzar características de bordes en múltiples direcciones, integrando estos mapas de bordes con la imagen original como entrada al modelo. Esta estrategia permite preservar estructuras finas y contornos, superando la tendencia al sobresuavizado observada en métodos previos.

La arquitectura de EDCNN se basa en una red completamente convolucional con conexiones densas, inspirada en DenseNet, que facilita la fusión de información jerárquica y de bordes a lo largo de la red. Además, emplea una función de pérdida compuesta que combina el error cuadrático medio (MSE) con una pérdida perceptual multi-escala basada en ResNet-50, lo que favorece la similitud tanto a nivel de píxel como de estructuras visuales. 

Los resultados reportados en el dataset NIH AAPM-Mayo Clinic LDCT demuestran que EDCNN logra una reducción de ruido efectiva y una mejor preservación de detalles en comparación con modelos clásicos y otros métodos de aprendizaje profundo, obteniendo altas puntuaciones en métricas cuantitativas (PSNR, SSIM) y en evaluaciones subjetivas realizadas por radiólogos. Este enfoque representa un avance significativo en el estado del arte para la mejora de imágenes CT de baja dosis, equilibrando la reducción de ruido y la conservación de información diagnóstica relevante.

\subsubsection{LEARN++}

Un avance relevante en el campo de la reconstrucción de imágenes de CT con sensado comprimido es el modelo LEARN++ \cite{LEARN++}. Esta arquitectura, basada en redes neuronales recurrentes de doble dominio, está diseñada para abordar los desafíos asociados a la reconstrucción a partir de un número reducido de vistas y a la reducción de la dosis de radiación. A diferencia de métodos tradicionales y enfoques basados en un solo dominio, LEARN++ procesa simultáneamente la información en los dominios de la imagen y del sinograma, permitiendo una interacción paralela y continua entre ambos. La red integra una subred convolucional dedicada a la restauración de imágenes y otra orientada al "inpainting" adaptativo de sinogramas, logrando así una mayor consistencia de datos y una mejor preservación de detalles anatómicos.

La función de pérdida compuesta de LEARN++ combina el error cuadrático medio tanto en el dominio de la imagen como en el sinograma, junto con una pérdida perceptual basada en características extraídas por VGG-19. Esta combinación permite equilibrar la fidelidad de los datos con la calidad visual y estructural de las imágenes reconstruidas. Los resultados obtenidos en el dataset NIH-AAPM-Mayo Clinic LDCT demuestran que LEARN++ supera significativamente a modelos previos en métricas cuantitativas como PSNR y SSIM, así como en evaluaciones subjetivas realizadas por radiólogos, destacándose por su capacidad para eliminar artefactos, reducir el ruido y preservar estructuras de bajo contraste.

En síntesis, LEARN++ representa un avance significativo en la reconstrucción de imágenes CT de baja dosis, estableciendo un marco generalizable que integra el procesamiento dual-domain y una función de pérdida compuesta. Su versatilidad y robustez lo posicionan como una alternativa eficaz tanto en escenarios clínicos con datos limitados como en aplicaciones de reducción de dosis, contribuyendo al desarrollo de técnicas más seguras y precisas en el ámbito de la imagenología médica.

\subsubsection{ULTRA}

El modelo ULTRA \cite{ULTRA} constituye una propuesta avanzada para la reconstrucción de imágenes en tomografía computarizada espectral (TC espectral) mediante aprendizaje profundo. Basado en una arquitectura U-net modificada con conexiones densas y filtros multicanal, ULTRA está diseñado para fusionar información multiescala y mejorar la extracción de características relevantes en imágenes adquiridas a diferentes energías. Entre sus innovaciones destaca la introducción de una función de pérdida generalizada $ L_p^p $, que permite controlar el equilibrio entre suavizado y preservación de bordes, así como una regularización por variación total anisotrópica que aprovecha las correlaciones espaciales y espectrales entre los distintos bins de energía.

Este enfoque aborda eficazmente los retos inherentes a la TC espectral, como la baja relación señal-ruido y la presencia de artefactos, superando las limitaciones de los métodos tradicionales basados en variación total o diccionarios tensoriales, que suelen ser computacionalmente costosos y sensibles a la selección de parámetros. ULTRA combina los beneficios del aprendizaje profundo con técnicas de regularización física, logrando una reconstrucción eficiente y precisa, con tiempos de procesamiento comparables a los métodos analíticos convencionales.

Los resultados experimentales, tanto en simulaciones como en estudios preclínicos y con fantomas físicos, demuestran que ULTRA supera a los métodos clásicos y otras redes profundas en métricas cuantitativas y cualitativas, preservando detalles anatómicos y mejorando la descomposición de materiales. Estas características posicionan a ULTRA como una solución robusta y escalable para aplicaciones clínicas futuras en TC espectral, con potencial para extenderse a geometrías tridimensionales y escenarios adversos.

\subsubsection{DLR}

Las técnicas de reconstrucción basadas en aprendizaje profundo (Deep Learning Reconstruction, DLR \cite{DLR}) han emergido como una alternativa avanzada para mejorar la calidad de imagen en angiografías por tomografía computarizada cerebral (CTA), superando las limitaciones de los métodos tradicionales como la retroproyección filtrada (FBP) y la reconstrucción iterativa híbrida (Hybrid IR). DLR utiliza redes neuronales convolucionales entrenadas con imágenes de referencia de alta calidad para diferenciar entre señal anatómica y ruido, permitiendo una reducción significativa del ruido y los artefactos sin sacrificar la resolución espacial ni la textura natural de la imagen.

Estudios recientes han demostrado que DLR no solo mejora métricas objetivas como la relación señal-ruido (SNR) y la relación contraste-ruido (CNR), sino que también incrementa la nitidez de bordes y la visualización de vasos pequeños, aspectos críticos en el diagnóstico de patologías vasculares intracraneales. Además, DLR ofrece tiempos de reconstrucción más rápidos que los métodos iterativos basados en modelos, lo que facilita su integración en la práctica clínica diaria. Algoritmos comerciales como AiCE (Canon) y TrueFidelity™ (GE) ya cuentan con validación clínica y aprobación regulatoria, consolidando el papel del aprendizaje profundo en la reconstrucción de imágenes médicas.

En síntesis, la reconstrucción basada en aprendizaje profundo representa un avance sustancial en la calidad y eficiencia de la CTA cerebral, permitiendo una mejor visualización de estructuras vasculares complejas y una reducción de artefactos, con potencial para optimizar el diagnóstico y tratamiento de enfermedades cerebrovasculares.

\section{Propuesta}

\subsection{Transformada Curvelet}

La transformada curvelet es una técnica de análisis multirresolución diseñada para representar de manera eficiente señales e imágenes con singularidades a lo largo de curvas suaves. A diferencia de la transformada wavelet, que ofrece una representación óptima de singularidades puntuales, la curvelet proporciona una representación \textbf{parsimoniosa} de estructuras curvilíneas y bordes en imágenes, gracias a su capacidad de adaptación direccional y anisotropía controlada\cite{Curvelets2000,FastCurveletTransform}.

\subsubsection{Fundamentos Matemáticos}

En su formulación continua, una curvelet se define como una función de base indexada por tres parámetros: 
\begin{itemize}
    \item \textbf{Escala} (\(j \in \mathbb{N}\)): Controla el tamaño de la curvelet.
    \item \textbf{Orientación} (\(\theta_l \in [0, 2\pi)\)): Determina la dirección principal de la curva.
    \item \textbf{Posición} (\(k \in \mathbb{Z}^2\)): Localiza la curvelet en el espacio.
\end{itemize}

La curvelet madre \(\phi_j(x)\) se dilata, rota y traslada para generar la familia de funciones:
\[
\phi_{j,l,k}(x) = 2^{-3j/4} \phi_j \left( R_{\theta_l}^{-1}(x - x_k^{(j,l)}) \right),
\]
donde \(R_{\theta_l}\) es la matriz de rotación y \(x_k^{(j,l)}\) denota la posición central en la escala \(j\) y orientación \(\theta_l\). En el dominio de Fourier, las curvelets se localizan en regiones \textbf{en forma de cuña}, con soporte anisotrópico que satisface la relación:
\[
\text{Ancho} \sim 2^{-j/2}, \quad \text{Largo} \sim 2^{-j}.
\]

\subsubsection{Transformada Curvelet Discreta}

La implementación discreta se realiza en el dominio de Fourier mediante los siguientes pasos\cite{FastCurveletTransform}:
\begin{enumerate}
    \item Aplicar la transformada de Fourier bidimensional (2D FFT) a la imagen.
    \item Multiplicar el espectro por ventanas angulares \(U_{j,l}(\omega)\) que aíslan bandas de frecuencia y dirección.
    \item Reorganizar (\emph{wrapping}) cada cuña espectral en un rectángulo centrado en el origen.
    \item Aplicar la transformada inversa de Fourier (2D IFFT) para obtener los coeficientes curvelet \(c(j,l,k)\).
\end{enumerate}

Matemáticamente, los coeficientes se calculan como:
\[
c(j,l,k) = \frac{1}{(2\pi)^2} \int_{\mathbb{R}^2} \hat{f}(\omega) U_{j,l}(\omega) e^{i\langle x_k^{(j,l)}, \omega \rangle} d\omega,
\]
donde \(\hat{f}(\omega)\) es el espectro de la imagen original.

\subsubsection{Ventajas sobre Otras Transformadas}

La curvelet supera a la wavelet en dos aspectos clave:
\begin{itemize}
    \item \textbf{Sensibilidad direccional}: Detecta bordes y curvas en múltiples orientaciones.
    \item \textbf{Representación esparsa}: Requiere menos coeficientes para representar edges, reduciendo redundancia.
\end{itemize}

Estas propiedades la hacen ideal para aplicaciones en imágenes médicas, donde la preservación de bordes anatómicos y la supresión de ruido son críticas.

\subsection{Transformada Synchrosqueezed Curvelet}

La transformada Synchrosqueezed Curvelet (SSCT) es una técnica avanzada de post-procesamiento que combina la capacidad direccional de la transformada curvelet con un método de reasignación espectral para lograr una representación más precisa de componentes modales en imágenes. Este enfoque es particularmente efectivo para analizar señales bidimensionales con frentes de onda curvos o componentes de banda estrecha, donde los métodos tradicionales fallan en separar modos superpuestos\cite{SynchrosqueezedCurveletTransform}.

\subsubsection{Principios Fundamentales}

La SSCT opera en dos etapas principales:
\begin{enumerate}
    \item \textbf{Transformada Curvelet Generalizada}: Aplica una transformada curvelet con parámetros de escalado geométrico adaptativos (escala radial \(t\) y angular \(s\)) para capturar componentes direccionales.
    \item \textbf{Reasignación Espectral (Synchrosqueezing)}: Reubica los coeficientes curvelet en el espacio fase basándose en estimaciones precisas de vectores de onda locales, condensando la energía en regiones más compactas.
\end{enumerate}

\subsubsection{Formulación Matemática}

Dada una imagen \(f(x)\), la SSCT se define mediante:
\begin{itemize}
    \item \textbf{Transformada Curvelet}: 
    \[
    W_f(a, \theta, b) = \langle f, \phi_{a,\theta,b} \rangle = \int_{\mathbb{R}^2} f(x) \overline{\phi_{a,\theta,b}(x)} dx
    \]
    donde \(\phi_{a,\theta,b}(x)\) son las curvelets con escala \(a\), orientación \(\theta\), y posición \(b\).
    
    \item \textbf{Estimación del Vector de Onda Local}:
    \[
    v_f(a, \theta, b) = \frac{\nabla_b \arg(W_f(a, \theta, b))}{2\pi}
    \]
    Este operador de fase estima la frecuencia instantánea en la dirección dominante.
    
    \item \textbf{Reasignación}:
    Los coeficientes se reubican según:
    \[
    T_f(v, b) = \int_{A(v, b)} W_f(a, \theta, b) a^{-3/2} da d\theta
    \]
    donde \(A(v, b) = \{(a, \theta): v_f(a, \theta, b) = v\}\) agrupa coeficientes con el mismo vector de onda estimado\cite{SynchrosqueezedCurveletTransform}.
\end{itemize}

\subsubsection{Ventajas Clave}

\begin{itemize}
    \item \textbf{Resolución Mejorada}: El synchrosqueezing reduce la dispersión espectral de los coeficientes curvelet, produciendo representaciones más nítidas (Fig. 2).
    \item \textbf{Separación de Modos}: Para señales \(f(x) = \sum_k f_k(x)\) con vectores de onda bien separados (\(|\nabla \phi_k - \nabla \phi_l| \geq d\)), la SSCT identifica cada componente \(f_k\) mediante clustering en el espacio fase reasignado.
    \item \textbf{Invariancia a la Curvatura}: A diferencia de métodos basados en wavelets, preserva la estructura de componentes curvilíneos gracias a la anisotropía de las curvelets.
\end{itemize}

\subsubsection{Aplicación en Procesamiento de Imágenes Médicas}

En el contexto de tomografías de cráneo, la SSCT permite:
\begin{itemize}
    \item Mejorar el contraste mediante la separación espectral de tejidos con diferentes propiedades de atenuación.
    \item Reducir artefactos de "blooming" en estructuras metálicas mediante la reasignación selectiva de coeficientes.
    \item Cuantificar texturas anatómicas mediante el análisis de mapas de vectores de onda locales.
\end{itemize}

\subsection{Propuesta Metodológica}

La propuesta desarrollada en esta tesis consiste en la aplicación de la transformada Synchrosqueezed Curvelet (SSCT) a imágenes de tomografía computarizada (CT) del cerebro. El procedimiento se inicia con la descomposición de la imagen original mediante la SSCT, obteniendo así una representación espectral detallada en el dominio espacio-frecuencia, capaz de capturar tanto las singularidades direccionales como la información multiescala inherente a las estructuras anatómicas cerebrales. Posteriormente, los coeficientes obtenidos a través de la SSCT son modificados mediante una función de procesamiento adecuada, diseñada para realzar las características de interés (como bordes o texturas) o atenuar componentes indeseados (como ruido o artefactos). Finalmente, se reconstruye la imagen a partir de los coeficientes modificados, generando una versión mejorada que se espera presente mayor calidad diagnóstica, con mejor contraste y preservación de detalles relevantes.

Cabe destacar que esta metodología es de naturaleza numérica pura y no depende de técnicas de inteligencia artificial ni de aprendizaje profundo. A diferencia de los métodos basados en redes neuronales, que han demostrado excelentes resultados en la mejora de imágenes médicas pero requieren grandes volúmenes de datos etiquetados, conocimien

\subsubsection{Thresholding}

El método de thresholding consiste en aplicar un umbral fijo a la energía obtenida por la SSCT, de modo que sólo se conserven los valores superiores a un cierto nivel predefinido. Este procedimiento es ampliamente utilizado en el procesamiento de imágenes médicas para segmentar regiones de interés o eliminar componentes de bajo valor energético, facilitando la extracción de estructuras relevantes \cite{zhao2023thresholding, pmc6132127}. La elección del valor de umbral es un parámetro crítico, ya que determina el equilibrio entre la preservación de detalles y la supresión de ruido o artefactos.

\subsubsection{Potenciación de la Energía SSCT}

La potenciación de la energía SSCT consiste en modificar los valores obtenidos de la transformada elevando la energía a una potencia específica (parámetro de enhancement). Esta operación no lineal permite realzar selectivamente las diferencias entre regiones de alta y baja energía, incrementando el contraste local y facilitando la discriminación de estructuras anatómicas sutiles. Métodos de potenciación y manipulación no lineal de coeficientes han sido explorados en el contexto de transformadas multiescala para el realce de imágenes médicas, mostrando mejoras en la percepción visual y en métricas objetivas de calidad \cite{SynchrosqueezedCurveletTransform, EnergyEnhancement}.

\subsubsection{Enmascaramiento del Resultado de la Transformada SSCT}

El enmascaramiento consiste en aplicar una máscara binaria o ponderada sobre el dominio SSCT, permitiendo conservar únicamente aquellas regiones que cumplen ciertos criterios (por ejemplo, localización anatómica, dirección predominante o magnitud de energía). Esta técnica es útil para focalizar el procesamiento en áreas de interés clínico y reducir la influencia de regiones irrelevantes o ruidosas. El enmascaramiento en el dominio de transformadas ha sido utilizado para mejorar la segmentación y el análisis de imágenes médicas, optimizando la relación señal-ruido y la especificidad de los resultados \cite{SynchrosqueezedCurveletTransform,ImageMaskingBook}.to experto para la curación de los conjuntos de entrenamiento y recursos computacionales avanzados (como procesadores gráficos de alto rendimiento), la propuesta aquí presentada puede implementarse con recursos computacionales convencionales y sin la necesidad de datos adicionales. Esta característica resulta especialmente ventajosa en contextos como el cubano, donde el acceso a infraestructuras de cómputo especializadas y bases de datos extensas es limitado.

\vspace{0.5cm}

En síntesis, la propuesta metodológica se fundamenta en el uso de herramientas matemáticas robustas y eficientes para el procesamiento de imágenes médicas, ofreciendo una alternativa viable y accesible para la mejora de la calidad de imágenes de CT cerebral en entornos con restricciones tecnológicas y de recursos humanos.

\chapter{Detalles de Implementación y Experimentos}\label{chapter:implementation}

En este capítulo se describe detalladamente el proceso de implementación del método propuesto para la mejora de imágenes de tomografía computarizada cerebral. Se presentan las herramientas y entornos de desarrollo empleados, así como las adaptaciones realizadas sobre las implementaciones existentes de la transformada synchrosqueezed (SST) y su inversa (ISST) en el contexto de la transformada curvelet bidimensional (2DCT). Además, se explican los procedimientos seguidos para el preprocesamiento de los datos, la configuración de los experimentos y la aplicación de las distintas estrategias de modificación de la matriz de energía. Finalmente, se justifica la selección de los parámetros experimentales y se expone la estructura general del flujo de trabajo, sentando las bases para la posterior evaluación y comparación de los resultados obtenidos.

\section{Ambiente de trabajo y herramientas}\label{section:work-environment}

El funcionamiento principal de los experimentos gira alrededor de la implementación original de las funciones de SST e ISST, por Haizhao Yang y Lexing Ying \cite{SynchrosqueezedCurveletTransform_SynLab}. Esta fue originalmente concebida en \texttt{Matlab}, un software y lenguaje de programación matemático, con gran énfasis en la practicidad y en la facilidad de uso para personal del sector científico. Esta implementación, con cambios mínimos, fue exportada a \texttt{GNU Octave}, una versión gratuita y de código abierto creada y mantenida por una extensa comunidad de contribuidores. Esto se hizo con el objetivo de poder utilizarlo sin la restricción de las barreras de pago, que en nuestro país se hace de gran dificultad. Esta implementación se encuentra modularizada en el repositorio de la tesis.

Dada la elevada dificultad de la implementación de SST y de ISST, además de la complejidad para crear experimentos complejos y paralelizables para mejor rendimiento, esta fue usada solo como una librería dinámica la cual fue llamada desde \texttt{Python 3.13}. Para esto, se usó la librería \texttt{oct2py}, la cual permite integración directa con el motor de \texttt{Octave} para llamar a las funciones definidas (en este caso solo SST e ISST) con objetos nativos de \texttt{Python}, como son arreglos n-dimensionales de \texttt{numpy}.

Durante el desarrollo y la ejecución de los experimentos, el código fue implementado y ejecutado íntegramente en la unidad central de procesamiento (CPU) de un equipo portátil de alto rendimiento. Este sistema cuenta con un procesador AMD Ryzen 7 8845HS de 8 núcleos y 16 hilos, con una frecuencia base de 3.8 GHz y una frecuencia máxima de hasta 5.1 GHz. Además, dispone de 32 GB de memoria RAM DDR5 y una unidad de almacenamiento sólido (SSD) de 1 TB. Todas las tareas de procesamiento, incluidas las etapas más intensivas computacionalmente, se realizaron exclusivamente en la CPU, sin recurrir a aceleración por hardware gráfico. Estas especificaciones proporcionaron un entorno adecuado para evaluar el rendimiento y la escalabilidad de la implementación propuesta, asegurando resultados reproducibles y tiempos de ejecución competitivos.

Adicionalmente, para el manejo, procesamiento y análisis de los resultados, se emplearon diversas herramientas del ecosistema científico de Python, tales como \texttt{numpy} para operaciones matriciales y de álgebra lineal, \texttt{matplotlib} para la visualización de datos y \texttt{pandas} para la manipulación de conjuntos de datos. El desarrollo de scripts experimentales, automatización de pruebas y gestión de resultados se realizó en un entorno de desarrollo basado en \texttt{Jupyter Notebook}, lo que facilitó la documentación interactiva y la reproducibilidad de los experimentos. Para el control de versiones y la colaboración, se utilizó el sistema \texttt{git}, permitiendo un seguimiento detallado de los cambios y la integración eficiente de mejoras en el código fuente. Finalmente, la ejecución de los experimentos se gestionó mediante scripts de automatización en \texttt{bash} y el uso de entornos virtuales con \texttt{conda} para asegurar la compatibilidad y portabilidad de las dependencias empleadas a lo largo del proyecto.

Para garantizar la reproducibilidad de los resultados y la consistencia en los experimentos, se fijaron explícitamente las semillas aleatorias y las configuraciones relevantes en todos los procesos. Esto permitió obtener resultados comparables y facilitar la validación de los procedimientos implementados.

\section{Optimización de la implementación}\label{section:optimization}

Con el objetivo de mejorar la eficiencia y el rendimiento de la solución propuesta, se han incorporado las siguientes optimizaciones en la implementación:

\begin{itemize}
    \item \textbf{Caché de los datos obtenidos:} Se ha implementado un sistema de almacenamiento temporal de los datos intermedios, como son el resultado de aplicar SST sobre una imagen en particular. Esta estrategia permite reutilizar resultados previamente calculados y reducir el acceso a la memoria principal o la necesidad de recomputar información, lo que se traduce en una disminución significativa del tiempo de ejecución global.
    
    \item \textbf{Paralelización de la ejecución:} Se ha aprovechado la capacidad de procesamiento concurrente de \texttt{Python} mediante la paralelización del procesamiento de imágenes independientes. Esto permite distribuir la carga de trabajo entre múltiples núcleos o hilos, acelerando etapas críticas como el cálculo de transformadas y la reconstrucción de imágenes, y logrando una mejora sustancial en los tiempos de cómputo.
    
    \item \textbf{Limpieza de los objetos no utilizados en memoria:} Para evitar la acumulación innecesaria de datos y prevenir posibles fugas de memoria, se ha incorporado un mecanismo de liberación sistemática de los objetos que ya no son requeridos durante la ejecución. Esta gestión eficiente de la memoria contribuye a mantener la estabilidad y el rendimiento de la aplicación, especialmente en experimentos de gran escala o en entornos con recursos limitados.
\end{itemize}

La necesidad de implementar optimizaciones en el procesamiento de imágenes surge de las exigencias computacionales observadas durante el desarrollo experimental. En primer lugar, el resultado de la SST aplicada a una sola imagen puede ocupar aproximadamente 1.4 GB de espacio en memoria, lo que representa una carga significativa para los recursos del sistema. Además, los tiempos de ejecución para el procesamiento completo de una imagen —incluyendo tanto la SST como su inversa (ISST)— alcanzan una media de 4 minutos, lo que limita la viabilidad del método en escenarios donde se requiere el análisis de grandes volúmenes de datos o la respuesta en tiempo real. Por otro lado, el consumo de memoria RAM sin la aplicación de optimizaciones puede llegar a ser muy intenso, alcanzando picos de hasta 15 GB durante la ejecución de múltiples experimentos. Estas condiciones motivan la incorporación de técnicas específicas de optimización, orientadas a reducir el uso de memoria, agilizar los tiempos de procesamiento y garantizar la estabilidad del sistema, permitiendo así la aplicación práctica y eficiente del método propuesto en entornos computacionales con recursos limitados.

Como resultado de las optimizaciones implementadas, se logró una reducción significativa tanto en los tiempos de ejecución como en el consumo de memoria del sistema. Tras la aplicación de estas mejoras, el tiempo medio de procesamiento por experimento —incluyendo la aplicación de SST e ISST sobre una imagen— disminuyó a aproximadamente 2 minutos, lo que representa una mejora sustancial en la eficiencia operativa del método. Asimismo, el uso máximo de memoria RAM durante la ejecución se redujo a 4 GB, permitiendo la realización de experimentos en equipos con recursos más limitados y mejorando la estabilidad general del sistema durante la ejecución de múltiples tareas en paralelo. Estos resultados evidencian la efectividad de las estrategias de optimización adoptadas en el desarrollo de la presente implementación.

\section{Implementación de los experimentos} \label{section:experiment-implementation}

La implementación de los experimentos se basa en un flujo modular que permite el procesamiento eficiente y reproducible de imágenes. El proceso inicia con la carga y normalización de las imágenes, que pueden provenir tanto de archivos individuales en formatos estándar (\texttt{.jpg} o \texttt{png}) como de volúmenes médicos (\texttt{.nii}). Para la gestión y muestreo de imágenes, se emplean objetos utilitarios que permiten recuperar conjuntos de imágenes desde directorios estructurados, facilitando la selección aleatoria o secuencial de muestras para los experimentos.

En cuanto al preprocesamiento, este se aplicó de manera diferenciada según la naturaleza de los experimentos. En una primera serie de experimentos, las imágenes fueron normalizadas para que sus valores de intensidad se encontraran en el rango $[0, 1]$. Esta normalización permitió asegurar la homogeneidad en la escala de entrada y facilitó la comparación directa entre diferentes muestras, independientemente de su origen o condiciones de adquisición.

En una segunda serie de experimentos, se utilizaron imágenes volumétricas en formato \texttt{.nii}, sobre las cuales se aplicó una ventana de intensidad específica para procesar únicamente la sección correspondiente a la densidad del cerebro. Esta técnica de ventana permitió resaltar las estructuras anatómicas de interés y reducir el impacto de regiones irrelevantes o valores atípicos, optimizando así la calidad y relevancia de los datos utilizados en el análisis posterior.

Cabe destacar que la naturaleza del preprocesamiento aplicado influyó notablemente en los tiempos de ejecución de los experimentos. En la primera serie, al trabajar con imágenes normalizadas en un rango reducido de valores, los tiempos de procesamiento fueron considerablemente menores, con una media aproximada de 1 minuto por experimento. En contraste, la segunda serie, basada en imágenes volumétricas procesadas mediante la aplicación de una ventana de intensidad, presentó tiempos de ejecución significativamente mayores, oscilando entre 3 y 5 minutos por experimento. Esta diferencia en el rendimiento también tuvo implicaciones directas en los resultados obtenidos, las cuales serán analizadas en detalle en una sección posterior.% todo

Una vez cargadas y preprocesadas, las imágenes son transformadas mediante la aplicación de la transformada synchrosqueezed curvelet y su inversa. Este procedimiento se encuentra encapsulado en objetos que gestionan tanto la ejecución de las transformadas como la utilización de un sistema de caché, el cual almacena los resultados intermedios para evitar recomputaciones innecesarias y optimizar el uso de recursos. Los resultados de la transformada se almacenan en estructuras especializadas que contienen la energía synchrosqueezed, los coeficientes curvelet y los vectores asociados, permitiendo su manipulación eficiente en las siguientes etapas del experimento.

Las funciones experimentales, están implementadas como procedimientos puros que operan sobre los resultados de la transformada, generando nuevas instancias modificadas de dichos resultados sin alterar los datos originales. La ejecución de cada experimento se gestiona a través de utilidades que reciben una configuración detallada, especificando la función experimental a aplicar, los parámetros y los argumentos adicionales requeridos. Tras la aplicación de la función experimental, se realiza la reconstrucción de la imagen modificada mediante la transformada inversa, asegurando la consistencia del flujo de procesamiento.

Durante todo el proceso, se implementa un sistema de registro para el seguimiento detallado de cada etapa y la gestión de errores. Además, se asegura la reproducibilidad de los experimentos mediante la fijación de semillas aleatorias y la serialización de los resultados y configuraciones empleadas. Este enfoque modular y estructurado permite realizar experimentos complejos de manera eficiente, garantizando la trazabilidad y la robustez de los resultados obtenidos.

El flujo completo de procesamiento experimental se organiza a través de una función principal que recibe como entrada un conjunto de imágenes, las configuraciones específicas de los experimentos a realizar y las configuraciones de las métricas de evaluación. Internamente, esta función itera sobre cada imagen y, para cada una, aplica de manera secuencial las distintas configuraciones experimentales definidas. Para cada combinación de imagen y configuración experimental, se ejecuta el procesamiento correspondiente: primero se realiza la transformada synchrosqueezed sobre la imagen, luego se aplica la función experimental seleccionada sobre los resultados de la transformada, y finalmente se reconstruye la imagen modificada mediante la transformada inversa.

Durante este proceso, se calculan las métricas de evaluación especificadas, comparando la imagen reconstruida con la original u otras referencias según corresponda. Todos los resultados, así como los parámetros y configuraciones utilizados, se almacenan y registran de forma estructurada para su posterior análisis. Este proceso automatizado permite ejecutar grandes volúmenes de experimentos de manera eficiente, garantizando la trazabilidad y la reproducibilidad (ver imagen \ref{fig:diagrama-flujo-experimentos}).

\begin{figure}[H]
    \centering
    \includegraphics[width=0.95\textwidth]{Graphics/diagrama experimentos tesis.drawio.png}
    \caption{Diagrama de flujo del pipeline experimental implementado para el procesamiento y análisis de imágenes.}
    \label{fig:diagrama-flujo-experimentos}
\end{figure}

\begin{table}[h]
    \centering
    \caption{Configuraciones experimentales y sus parámetros.\cite{ExperimentSource,ExperimentSource2}}
    \label{tab:experimentos}
    \begin{tabular}{>{\raggedright}p{4cm}p{6cm}}
    \toprule
    \textbf{Tipo de Experimento} & \textbf{Parámetros} \\ 
    \midrule
    Umbralizado & threshold = 0.1, 0.05, 0.001 \\
    \midrule
    Mejora de energía & \texttt{enhancement\_factor} = 3.0, 1.5, 10 \\
    \midrule
    Filtro pasa-alto & \texttt{cutoff\_scale} = 5, 10 \\
    \midrule
    Filtro pasa-banda & \texttt{low\_scale} = 5 (\texttt{high\_scale} = 10), 15 (20) \\
    \midrule
    Máscara gaussiana & $ \mu  = N/2$, $ \sigma $ = 1.0, 4.0, 6.0 \\
    \midrule
    Máscara exponencial & \texttt{decay\_rate} = 0.1, 0.01 \\
    \bottomrule
    \end{tabular}
    \footnotesize{\\*N/2: mitad de las escalas disponibles en la descomposición.}
\end{table}

Para la realización de los experimentos se utilizó un conjunto de configuraciones específicas, las cuales se detallan en la Tabla~\ref{tab:experimentos}. Cada configuración define los parámetros y valores empleados en las distintas funciones experimentales aplicadas durante el procesamiento, permitiendo así evaluar el comportamiento del método bajo diferentes condiciones y ajustes. Esta variedad de configuraciones facilita un análisis exhaustivo y comparativo de los resultados obtenidos.

\section{Implementación de las métricas} \label{section:metrics-implementation}

Las métricas utilizadas para la evaluación cuantitativa de los resultados experimentales fueron implementadas de manera modular y eficiente, empleando herramientas especializadas del ecosistema científico de \texttt{Python}. Las configuraciones específicas de las métricas se encuentran detalladas en la Tabla~\ref{tab:metricas}

Para las métricas basadas en operaciones numéricas y estadísticas, como el índice de mejora de contraste y la razón de nitidez basada en el operador Laplaciano, se utilizó la biblioteca \texttt{numpy} para el cálculo vectorizado de desviaciones estándar y varianzas, así como \texttt{opencv-python} para la aplicación de operadores de filtrado espacial.

Las métricas perceptuales avanzadas, como el FSIM (Feature Similarity Index) y LPIPS (Learned Perceptual Image Patch Similarity), se implementaron a través de las bibliotecas \texttt{piq} y \texttt{lpips}, respectivamente, haciendo uso de \texttt{PyTorch} para el manejo de tensores y la ejecución eficiente en CPU. El índice de similitud estructural (SSIM) y la relación señal-ruido pico (PSNR) se calcularon utilizando las funciones \texttt{structural\_similarity} y \texttt{peak\_signal\_noise\_ratio} de la biblioteca \texttt{scikit-image}. Para la métrica BRISQUE, orientada a la evaluación de artefactos y ruido perceptual, se empleó la implementación disponible en \texttt{piq}.

El flujo de cálculo de métricas se integra al final de cada experimento: una vez reconstruida la imagen modificada, se aplican de forma automática todas las métricas relevantes, comparando la imagen procesada con la original o con una referencia apropiada (en este caso, con la imagen original). Cada función de métrica está desacoplada del resto del procesamiento y recibe como entrada los arreglos de imagen en formato \texttt{numpy}, asegurando así su reutilización y extensibilidad.

Los resultados de cada métrica, junto con la configuración experimental correspondiente, son almacenados y registrados de forma estructurada al finalizar cada experimento. Este enfoque garantiza la trazabilidad y facilita el análisis comparativo entre diferentes configuraciones y tipos de preprocesamiento. Además, la modularidad de la implementación permite la fácil incorporación de nuevas métricas en futuras extensiones del trabajo.

De manera concreta, para cada imagen procesada, una vez ejecutados todos los experimentos definidos, se calculan de forma automática todas las métricas seleccionadas para cada resultado. Posteriormente, los valores obtenidos de las métricas, junto con la información sobre la imagen y la configuración experimental utilizada, son serializados y almacenados en un archivo en formato \texttt{json}. Esta estrategia permite centralizar y organizar los datos de evaluación, facilitando su análisis posterior, la comparación sistemática entre diferentes métodos y la integración con herramientas externas de análisis estadístico o visualización.

\begin{table}[h]
    \centering
    \caption{Métricas de evaluación agrupadas por categorías\cite{Metrics}}
    \label{tab:metricas}
    \begin{tabular}{p{3cm}p{8cm}p{3cm}}
    \toprule
    \textbf{Categoría} & \textbf{Métrica} & \textbf{Descripción} \\ 
    \midrule
    \multirow{3}{*}{Mejora} 
    & CII (\textit{Contrast Improvement Index}) & Índice de mejora de contraste \\
    & LSR (\textit{Laplacian Sharpness Ratio}) & Cociente de nitidez Laplaciana \\
    & FSIM (\textit{Feature Similarity Index}) & Similaridad de características \\
    \midrule

    \multirow{3}{*}{Distorsión}
    & PSNR (\textit{Peak Signal-to-Noise Ratio}) & Relación señal-ruido \\
    & SSIM (\textit{Structural Similarity Index}) & Similaridad estructural \\
    & LPIPS (\textit{Learned Perceptual Image Patch Similarity}) & Similaridad perceptual \\
    \midrule

    \multirow{2}{*}{Artefactos}
    & RNS (Residual Noise Standard Deviation) & Desviación estándar del ruido residual \\
    & BRISQUE (Blind/Referenceless Image Spatial Quality Evaluator \cite{BRISQUE}) & Evaluador ciego de calidad \\
    \bottomrule
    \end{tabular}
\end{table}

\section{Implementación de las estadísticas}\label{section:statistics-implementation}

El análisis estadístico de los resultados experimentales se diseñó para comparar de manera rigurosa el desempeño de los distintos métodos evaluados. Para cada par de experimentos a comparar, se recopilaron los valores obtenidos de las métricas correspondientes en todas las imágenes procesadas. El primer paso consistió en la aplicación de pruebas de normalidad sobre la distribución de cada métrica, con el objetivo de determinar el tipo de análisis estadístico adecuado para la comparación. Las pruebas de normalidad empleadas incluyeron, entre otras, la prueba de Shapiro-Wilk.

En función de los resultados de la prueba de normalidad, se seleccionó el método estadístico más apropiado para la comparación de los dos grupos experimentales. Cuando la distribución de las métricas fue compatible con la normalidad, se utilizó la prueba t de Student para muestras independientes, permitiendo evaluar si existían diferencias estadísticamente significativas entre las medias de los grupos. En caso contrario, es decir, si la distribución no era normal, se recurrió a pruebas no paramétricas como la prueba de Mann-Whitney U, que permite comparar la mediana de dos muestras independientes sin asumir normalidad.

Si el análisis estadístico indicó que existían diferencias significativas entre los experimentos comparados, se procedió a comparar sus medias y desviaciones estándar para cada métrica, determinando así cuál de los dos experimentos presentaba un mejor desempeño según los valores observados. En caso de que no se detectaran diferencias estadísticamente significativas, se asumió que ambos experimentos ofrecían resultados equivalentes para la métrica analizada.

\begin{table}[h]
    \centering
    \caption{Resumen de las métricas utilizadas, sus valores ideales y rangos de referencia.}
    \label{tab:metrics-expected-values}
    \begin{tabular}{>{\raggedright}p{5cm}p{3cm}p{3cm}}
        \toprule
        \textbf{Métrica} & \textbf{Valor ideal} & \textbf{Rango} \\
        \midrule
        Índice de mejora de contraste (CII) & $>1$ & $[0, \infty)$ \\
        Razón laplaciana & $>1$ & $[0, \infty)$ \\
        FSIM & $1$ & $[0, 1]$ \\
        PSNR & $\infty$ & $[0, \infty)$ \\
        SSIM & $1$ & $[0, 1]$ \\
        LPIPS & $0$ & $[0, 1]$ \\
        Desviación estándar del ruido residual & $0$ & $[0, \infty)$ \\
        BRISQUE & $0$ & $[0, 100]$ \\
        \bottomrule
    \end{tabular}
\end{table}

Adicionalmente, para cada métrica se consultó una tabla de valores esperados o ideales (Tabla~\ref{tab:metrics-expected-values}), lo que permitió contextualizar los resultados obtenidos y seleccionar el experimento más favorable dentro de cada par comparado. En este proceso, se otorgó un peso mayor a aquellas métricas consideradas prioritarias para la calidad de la imagen, como el índice de similitud estructural (SSIM), que evalúa la preservación de la estructura, y la métrica de nitidez basada en el operador Laplaciano, que prioriza la definición de bordes. Esta ponderación permitió orientar la selección final hacia los experimentos que maximizan la fidelidad estructural y la nitidez en las imágenes reconstruidas.

Para el análisis estadístico de los resultados experimentales se emplearon principalmente las bibliotecas \texttt{pandas} para la manipulación y análisis de los datos tabulados, y \texttt{scipy.stats} para la ejecución de pruebas estadísticas como el test de normalidad y las pruebas de comparación entre grupos. Adicionalmente, las bibliotecas \texttt{seaborn} y \texttt{matplotlib.pyplot} se utilizaron para la visualización gráfica de los resultados y la exploración de tendencias y distribuciones. Estas herramientas permitieron realizar un análisis estadístico riguroso, reproducible y visualmente comprensible de los datos obtenidos en los experimentos.

\chapter{Resultados}\label{chapter:results}

En este capítulo se presentan y analizan los resultados obtenidos a partir de la aplicación de los experimentos propuestos en capítulos anteriores (ver epígrafe ~\ref{section:experiment-implementation}). Se exponen las métricas estadísticas calculadas para cada grupo de experimentos, identificando las configuraciones que han demostrado el mejor desempeño. Asimismo, se realiza una comparación entre los resultados más destacados de cada grupo experimental. Finalmente, se incluye una evaluación cualitativa realizada por una especialista del área médica, complementada con el correspondiente análisis estadístico, con el objetivo de validar la relevancia y aplicabilidad de los hallazgos en el contexto clínico.

\section{Mejores resultados por grupos de experimentos}

\subsection{Mejora de energía}

En todos los experimentos realizados empleando la técnica de mejora de energía (véase la Tabla~\ref{tab:metricas}), no se observaron diferencias estadísticamente significativas entre los resultados obtenidos ($p > 0.05$). Esta ausencia de significancia se mantiene tanto al aplicar el preprocesamiento de normalización como al utilizar la extracción de ventana de valores. 

Sin embargo, se seleccionó el resultado correspondiente al parámetro de mejora de energía igual a 10, ya que esta configuración presentó los mejores tiempos de ejecución entre todas las evaluadas, lo que resulta especialmente relevante para aplicaciones prácticas donde la eficiencia computacional es prioritaria.

\subsection{Umbralizado}

En todos los experimentos realizados empleando la técnica de umbralizado (véase la Tabla~\ref{tab:metricas}), no se observaron diferencias estadísticamente significativas entre los resultados obtenidos ($p > 0.05$). Esta ausencia de significancia se mantiene tanto al aplicar el preprocesamiento de normalización como al utilizar la extracción de ventana de valores.

Sin embargo, se seleccionó el resultado correspondiente al umbral de 0.001, ya que esta configuración presentó los mejores tiempos de ejecución entre todas las evaluadas, lo que resulta especialmente relevante para aplicaciones prácticas donde la eficiencia computacional es prioritaria.

\subsection{Máscara de pasa alto}

Para evaluar el impacto de los diferentes parámetros en el filtrado de pasa alto, se empleó la métrica BRISQUE, la cual permite estimar la calidad de imagen de manera automática y sin referencia, asignando puntuaciones donde valores más bajos indican mejor calidad percibida\cite{BRISQUE}. En este contexto, se compararon dos configuraciones del filtro pasa alto, utilizando valores de \texttt{cutoff\_scale} de 5 y 10.

Los resultados obtenidos muestran que la configuración con \texttt{cutoff\_scale} igual a 5 alcanzó una puntuación BRISQUE de 108.93, mientras que la configuración con \texttt{cutoff\_scale} igual a 10 obtuvo un valor de 110.90. Dado que una menor puntuación BRISQUE indica una menor presencia de artefactos y, por tanto, una mejor calidad de imagen, se considera que la configuración con \texttt{cutoff\_scale} de 5 presenta un desempeño superior según esta métrica.

Cabe destacar que, aunque se evaluaron otras métricas de calidad, no se encontró suficiente evidencia estadística para afirmar que existan diferencias significativas entre ambas configuraciones en dichas métricas. Por lo tanto, la selección de la configuración óptima se fundamenta en los resultados obtenidos con BRISQUE, en concordancia con los criterios objetivos de evaluación empleados en este trabajo.

\subsection{Máscara de pasa banda}

Para analizar el efecto de los parámetros en el filtrado Bandpass, se empleó la métrica Laplacian Sharpness Ratio (LSR), que cuantifica el nivel de nitidez en la imagen mediante el análisis de bordes y detalles finos, siendo un indicador clave de la calidad visual en imágenes biomédicas. En este análisis, se compararon dos configuraciones de la máscara de pasa banda: \texttt{low\_scale=5, high\_scale=10} y \texttt{low\_scale=15, high\_scale=20}.

Los resultados indican que la configuración \textit{Máscara de pasa banda con low\_scale=5, high\_scale=10} obtuvo una puntuación LSR de 54.92, mientras que la configuración con \texttt{low\_scale=15, high\_scale=20} alcanzó un valor de 37.54. Dado que un mayor valor de LSR refleja una mayor nitidez y, por ende, mejor calidad visual, se concluye que la primera configuración ofrece un desempeño superior según esta métrica.

Es importante señalar que, aunque se evaluaron otras métricas de calidad, no se encontró suficiente evidencia estadística para afirmar que existan diferencias significativas entre ambas configuraciones en dichas métricas. Por consiguiente, la selección de la configuración óptima se fundamenta en los resultados obtenidos con LSR, en línea con los criterios objetivos de evaluación adoptados en este estudio.

\subsection{Máscara de escala exponencial}

En el análisis de la máscara de escala exponencial, se compararon dos configuraciones experimentales con valores de \texttt{decay\_rate} de 0.1 y 0.01. Los resultados muestran que ambas configuraciones son equivalentes en términos de las métricas de calidad evaluadas, ya que no se encontró suficiente evidencia estadística para afirmar que sus resultados sean diferentes.

Dado que la configuración con \texttt{decay\_rate} igual a 0.01 presentó un mejor desempeño en tiempos de ejecución, se selecciona esta opción como la más adecuada para aplicaciones donde la eficiencia computacional es prioritaria.

\subsection{Máscara de escala gaussiana}

Se evaluaron tres configuraciones de la máscara de escala gaussiana, variando el parámetro \texttt{sigma} en valores de 1.0, 4.0 y 6.0, manteniendo \texttt{center\_scale=N/2}. Los resultados muestran que las configuraciones con \texttt{sigma=4.0} y \texttt{sigma=6.0} son equivalentes en términos de las métricas de calidad evaluadas, sin diferencias estadísticamente significativas.

Sin embargo, la configuración con \texttt{sigma=1.0} superó a las demás en la mayoría de las métricas relevantes, incluyendo CII, LSR, PSNR, SSIM, LPIPS y RNS, lo que indica una mejor calidad de imagen y menor distorsión. Por tanto, se selecciona la máscara de escala gaussiana con \texttt{center\_scale=N/2, sigma=1.0} como la opción óptima para este grupo experimental, en concordancia con los criterios objetivos de evaluación empleados en este trabajo.

\begin{table}[H]
    \centering
    \caption{Comparativa de métricas para las configuraciones de la máscara de escala gaussiana (\texttt{center\_scale=N/2})}
    \begin{tabular}{|l|c|c|c|}
    \hline
    \textbf{Métrica} & \textbf{sigma=1.0} & \textbf{sigma=4.0} & \textbf{sigma=6.0} \\
    \hline
    CII & 1.4690 & 1.2515 & 1.2515 \\
    LSR & 67.1471 & 19.2774 & 19.2774 \\
    PSNR & 65.7315 & 63.7677 & 63.7677 \\
    SSIM & 0.9976 & 0.9963 & 0.9963 \\
    LPIPS & 0.00021 & 0.00036 & 0.00036 \\
    RNS & 0.1129 & 0.1405 & 0.1405 \\
    BRISQUE & 111.96 & 106.81 & 106.81 \\
    \hline
    \end{tabular}
    \label{tab:gaussian_comparativa}
\end{table}

\section{Selección del mejor modelo}

El proceso de selección del mejor modelo se realizó a partir de una rigurosa comparación entre los experimentos que obtuvieron los mejores resultados en cada una de las categorías evaluadas: umbralizado, máscaras exponenciales y gaussianas, filtros pasa alto y pasa banda, así como la mejora de energía. Para garantizar una evaluación objetiva y exhaustiva, se emplearon múltiples métricas de calidad de imagen, incluyendo LSR (Laplacian Sharpness Ratio), BRISQUE, PSNR, SSIM, LPIPS y RNS, cubriendo aspectos de nitidez, distorsión, artefactos y similitud estructural.

En la primera etapa, se identificaron los experimentos con mejor desempeño dentro de cada grupo, seleccionando los siguientes: \textit{Threshold Coefficients at 0.001}, \textit{Exponential Scale Mask with decay\_rate=0.01}, \textit{Highpass Mask with cutoff\_scale=10}, \textit{Gaussian Scale Mask with center\_scale=N/2, sigma=1.0}, \textit{Bandpass Mask with low\_scale=5, high\_scale=10} y \textit{Enhance Energy by 10}.

Posteriormente, se realizó una comparación cruzada entre estos modelos. Los resultados mostraron que varias configuraciones, como el umbralizado y la mejora de energía, presentaron desempeños equivalentes en la mayoría de las métricas evaluadas, sin diferencias estadísticamente significativas. Sin embargo, a medida que se avanzó en la comparación directa con los otros métodos, emergieron diferencias claras en métricas clave.

La máscara gaussiana con \texttt{center\_scale=N/2, sigma=1.0} destacó consistentemente sobre las demás configuraciones. Superó a los otros modelos en métricas como LSR (67.15 frente a valores significativamente menores en otros métodos), PSNR (65.73), SSIM (0.9976), LPIPS (0.00021), y RNS (0.1129), lo que indica una mayor nitidez, menor distorsión y mejor preservación de la estructura y los detalles de la imagen. Además, aunque en la métrica BRISQUE no obtuvo el valor más bajo absoluto, su desempeño global en el conjunto de métricas fue superior.

Este proceso de selección evidencia la importancia de considerar múltiples dimensiones de calidad de imagen y no basar la decisión únicamente en una métrica aislada. Así, la \textit{Máscara de escala gaussiana con center\_scale=N/2, sigma=1.0} fue seleccionada como el mejor modelo global, al ofrecer el balance óptimo entre nitidez, preservación de detalles y baja distorsión, validando su aplicabilidad en el procesamiento avanzado de imágenes de CT cerebral.

\begin{table}[H]
\centering
\caption{Comparativa de métricas para los mejores modelos de cada categoría}
\resizebox{\textwidth}{!}{
\begin{tabular}{|l|c|c|c|c|c|c|}
\hline
\textbf{Métrica} & \textbf{Umbral 0.001} & \textbf{Exp. 0.01} & \textbf{Pasa alto 10} & \textbf{Gaussiana 1.0} & \textbf{Pasa banda 5-10} & \textbf{Energía 10} \\
\hline
CII & 1.2515 & 1.2515 & -- & 1.4690 & 1.2515 & 1.2515 \\
LSR & 43.41 & 43.41 & 54.92 & 67.15 & 54.92 & 43.41 \\
PSNR & -- & -- & -- & 65.73 & 63.77 & -- \\
SSIM & 0.9972 & 0.9963 & 0.9968 & 0.9976 & 0.9963 & 0.9963 \\
LPIPS & 0.00026 & 0.00036 & 0.00030 & 0.00021 & 0.00036 & 0.00036 \\
RNS & 0.1235 & 0.1405 & 0.1308 & 0.1129 & 0.1405 & 0.1405 \\
BRISQUE & 110.90 & 106.81 & 108.99 & 111.96 & 106.81 & 106.81 \\
\hline
\end{tabular}
}
\label{tab:comparativa_final}
\end{table}

\section{Evaluación cualitativa por especialista}

Con el objetivo de complementar el análisis cuantitativo y validar la relevancia clínica de los métodos propuestos, se realizó una evaluación cualitativa a través de una encuesta dirigida a una especialista en radiología. Para este fin, se seleccionaron imágenes mejoradas utilizando el mejor método identificado en la comparación cuantitativa (máscara gaussiana) y el método con menor desempeño (umbralizado).

La encuesta consistió en 32 pares de imágenes, cada par compuesto por una imagen mejorada con la máscara gaussiana y otra con el método de umbralizado. Los pares fueron seleccionados aleatoriamente de distintas secciones del conjunto de datos, y el orden de presentación fue completamente aleatorio para evitar cualquier sesgo y garantizar que la especialista no pudiera identificar el método aplicado en cada caso.

Los resultados obtenidos revelaron que, en 19 casos, ambos métodos fueron considerados equivalentes por la evaluadora. En 9 casos, el método de umbralizado fue valorado como “mucho mejor” y, en 3 casos, la máscara gaussiana recibió dicha calificación.

Para analizar estadísticamente estas preferencias, se aplicó la prueba de McNemar, adecuada para comparar proporciones en datos pareados y determinar si existen diferencias significativas en la preferencia entre dos métodos\cite{mcnemar}. El valor del estadístico de McNemar fue 2.0833, con un valor-p de 0.1489. Según estos resultados, aunque se observó una preferencia por el método de umbralizado en los casos discordantes (50.0\% de diferencia), esta diferencia no alcanzó significancia estadística al nivel convencional ($p > 0.05$).

Estos hallazgos sugieren que, desde la perspectiva cualitativa de la especialista, no existe una diferencia significativa en la percepción de calidad entre ambos métodos en el contexto evaluado.

\backmatter

\begin{conclusions}
    El presente trabajo de diploma tuvo como objetivo general desarrollar un método numérico basado en la transformada Synchrosqueezed aplicada a Curvelets (SST-2DCT) para el mejoramiento del contraste en imágenes de tomografía computarizada (CT) de cráneo, con especial énfasis en la visualización de tejidos blandos y lesiones pequeñas sin necesidad de utilizar agentes de contraste radiológico. Este enfoque se planteó como una alternativa viable en contextos con restricciones de recursos computacionales y limitaciones en el acceso a grandes volúmenes de datos etiquetados, como es el caso del sistema de salud cubano.

    \bigskip

    A continuación, se exponen las principales conclusiones obtenidas, organizadas en correspondencia con los objetivos específicos propuestos:

    \begin{enumerate}
        \item \textbf{Implementación de la transformada SST-2DCT e ISST en Python 3.13:} Se logró desarrollar una implementación funcional y eficiente de la transformada Synchrosqueezed Curvelet y su inversa, adaptando herramientas de Octave mediante la interfaz \textit{oct2py}. Esta integración permitió la manipulación directa de imágenes en formato NIfTI y facilitó la ejecución de experimentos complejos en entornos de cómputo de propósito general.

        \item \textbf{Diseño y ejecución de experimentos de mejora de características de imagen:} Se llevaron a cabo múltiples experimentos que aplicaron diferentes estrategias de modificación sobre la matriz de energía SST, tales como potenciación, umbralización y enmascaramiento. Entre todas las variantes evaluadas, la técnica basada en \textbf{Máscara Gaussiana con parámetros $\mu = N/2$ y $\delta = 1$} resultó ser la más adecuada, al lograr un equilibrio favorable entre el realce de estructuras anatómicas sutiles y la preservación de la calidad global de la imagen.

        \item \textbf{Evaluación cuantitativa mediante métricas objetivas:} Los resultados fueron evaluados utilizando un conjunto de métricas agrupadas en tres categorías: mejora de contraste (CII, LSR, FSIM), distorsión (PSNR, SSIM, LPIPS) y detección de artefactos (RNS, BRISQUE). La \textbf{Máscara Gaussiana seleccionada} demostró un desempeño superior en la mejora del contraste y la definición de bordes sin introducir distorsiones o artefactos visuales significativos, manteniendo la integridad estructural de las imágenes.

        \item \textbf{Comparación con métodos clásicos y modernos de mejoramiento de imágenes:} La solución propuesta mostró resultados competitivos frente a métodos tradicionales como el filtrado gaussiano, la ecualización de histograma y transformaciones homomórficas, superándolos en la preservación de detalles finos y en la reducción de artefactos. No obstante, en comparación con métodos basados en inteligencia artificial, la propuesta ofrece menores prestaciones en escenarios donde se dispone de grandes volúmenes de datos entrenados, aunque presenta la ventaja de no requerir entrenamiento previo ni bases de datos anotadas.

        \item \textbf{Análisis de viabilidad en entornos con recursos limitados:} La solución desarrollada demostró ser viable para su implementación en contextos con restricciones computacionales, debido a su baja dependencia de hardware especializado y su menor complejidad en comparación con las técnicas de aprendizaje profundo. Sin embargo, \textbf{la principal limitación identificada radica en los tiempos de ejecución}, que oscilan entre \textbf{2 y 7 minutos por imagen} dependiendo de si la transformada SST ya ha sido calculada previamente o si debe computarse nuevamente desde cero, aspecto que podría representar una restricción en aplicaciones clínicas de tiempo real.
    \end{enumerate}

    \bigskip

    \textbf{Conclusión general:} El método propuesto basado en la transformada Synchrosqueezed Curvelet constituye una alternativa efectiva y accesible para el mejoramiento del contraste en imágenes de tomografía computarizada de cráneo, especialmente útil en situaciones donde no es posible el uso de agentes de contraste o tecnologías de inteligencia artificial avanzadas. Los resultados obtenidos evidencian que es posible realzar detalles anatómicos clínicamente relevantes, como lesiones pequeñas y estructuras de bajo contraste, sin introducir artefactos visuales significativos ni comprometer la integridad de la imagen original.

    En términos de impacto, la metodología desarrollada ofrece un marco numérico reproducible que puede ser utilizado como etapa previa al entrenamiento de modelos de aprendizaje profundo, contribuyendo a la reducción de la complejidad de estos modelos y potenciando su capacidad de generalización. Asimismo, sienta las bases para el desarrollo de sistemas híbridos que combinen las ventajas de los métodos numéricos y de la inteligencia artificial, aportando interpretabilidad y transparencia al proceso de diagnóstico asistido por computadora.

    Finalmente, se reconoce que la técnica propuesta aún presenta limitaciones, como la necesidad de ajuste manual de parámetros experimentales y los tiempos de procesamiento relativamente elevados para cada imagen, aspectos que deberán ser abordados en trabajos futuros con vistas a mejorar su aplicabilidad en entornos clínicos exigentes.
\end{conclusions}

\begin{recomendations}

    A partir de los resultados obtenidos y las limitaciones identificadas durante el desarrollo de esta investigación, se proponen las siguientes recomendaciones para trabajos futuros:

    \begin{enumerate}
    \item \textbf{Implementación nativa de SST e ISST en Python:} Se recomienda desarrollar una versión completa de los algoritmos de SST e ISST directamente en Python. Esto permitiría aprovechar capacidades avanzadas de procesamiento paralelo, tanto a nivel de CPU como de GPU, con el fin de reducir de manera significativa los tiempos de ejecución observados y facilitar su integración en entornos clínicos o de investigación que requieren procesamiento en tiempo real.

    \item \textbf{Exploración de otros métodos numéricos:} Se sugiere investigar y evaluar nuevas técnicas numéricas para la mejora de imágenes médicas mediante la metodología experimental establecida en este trabajo. Este esfuerzo podría conducir a la identificación de transformadas o algoritmos alternativos que proporcionen mejoras adicionales en la calidad de imagen o en la eficiencia computacional, sin incurrir en la complejidad y requisitos de datos de los enfoques basados en inteligencia artificial.

    \item \textbf{Desarrollo de métodos híbridos basados en IA y SST:} Se propone crear esquemas híbridos que combinen la transformada Synchrosqueezed Curvelet con modelos de inteligencia artificial. En particular, se recomienda entrenar mediante aprendizaje por refuerzo un modelo capaz de predecir la máscara óptima que se aplicaría sobre la energía obtenida tras la SST de una imagen determinada. Este enfoque presenta ventajas potenciales como la eliminación de la necesidad de datos etiquetados para el entrenamiento y una baja exigencia computacional, debido al reducido tamaño de las máscaras utilizadas. Aunque esta propuesta aún se considera hipotética, representa una línea de investigación prometedora que puede mejorar la adaptabilidad y el rendimiento del método desarrollado.
    \end{enumerate}
\end{recomendations}

\printbibliography[heading=bibintoc, title={Referencias Bibliográficas}]


\end{document}