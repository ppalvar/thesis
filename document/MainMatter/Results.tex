\chapter{Resultados}\label{chapter:results}

En este capítulo se presentan y analizan los resultados obtenidos a partir de la aplicación de los experimentos propuestos en capítulos anteriores (ver epígrafe ~\ref{section:experiment-implementation}). Se exponen las métricas estadísticas calculadas para cada grupo de experimentos, identificando las configuraciones que han demostrado el mejor desempeño. Asimismo, se realiza una comparación entre los resultados más destacados de cada grupo experimental. Finalmente, se incluye una evaluación cualitativa realizada por una especialista del área médica, complementada con el correspondiente análisis estadístico, con el objetivo de validar la relevancia y aplicabilidad de los hallazgos en el contexto clínico.

\section{Mejores resultados por grupos de experimentos}

\subsection{Mejora de energía}

En todos los experimentos realizados empleando la técnica de mejora de energía (véase la Tabla~\ref{tab:metricas}), no se observaron diferencias estadísticamente significativas entre los resultados obtenidos ($p > 0.05$). Esta ausencia de significancia se mantiene tanto al aplicar el preprocesamiento de normalización como al utilizar la extracción de ventana de valores. 

Sin embargo, se seleccionó el resultado correspondiente al parámetro de mejora de energía igual a 10, ya que esta configuración presentó los mejores tiempos de ejecución entre todas las evaluadas, lo que resulta especialmente relevante para aplicaciones prácticas donde la eficiencia computacional es prioritaria.

\subsection{Umbralizado}

En todos los experimentos realizados empleando la técnica de umbralizado (véase la Tabla~\ref{tab:metricas}), no se observaron diferencias estadísticamente significativas entre los resultados obtenidos ($p > 0.05$). Esta ausencia de significancia se mantiene tanto al aplicar el preprocesamiento de normalización como al utilizar la extracción de ventana de valores.

Sin embargo, se seleccionó el resultado correspondiente al umbral de 0.001, ya que esta configuración presentó los mejores tiempos de ejecución entre todas las evaluadas, lo que resulta especialmente relevante para aplicaciones prácticas donde la eficiencia computacional es prioritaria.

\subsection{Máscara de pasa alto}

Para evaluar el impacto de los diferentes parámetros en el filtrado de pasa alto, se empleó la métrica BRISQUE, la cual permite estimar la calidad de imagen de manera automática y sin referencia, asignando puntuaciones donde valores más bajos indican mejor calidad percibida\cite{BRISQUE}. En este contexto, se compararon dos configuraciones del filtro pasa alto, utilizando valores de \texttt{cutoff\_scale} de 5 y 10.

Los resultados obtenidos muestran que la configuración con \texttt{cutoff\_scale} igual a 5 alcanzó una puntuación BRISQUE de 108.93, mientras que la configuración con \texttt{cutoff\_scale} igual a 10 obtuvo un valor de 110.90. Dado que una menor puntuación BRISQUE indica una menor presencia de artefactos y, por tanto, una mejor calidad de imagen, se considera que la configuración con \texttt{cutoff\_scale} de 5 presenta un desempeño superior según esta métrica.

Cabe destacar que, aunque se evaluaron otras métricas de calidad, no se encontró suficiente evidencia estadística para afirmar que existan diferencias significativas entre ambas configuraciones en dichas métricas. Por lo tanto, la selección de la configuración óptima se fundamenta en los resultados obtenidos con BRISQUE, en concordancia con los criterios objetivos de evaluación empleados en este trabajo.

\subsection{Máscara de pasa banda}

Para analizar el efecto de los parámetros en el filtrado Bandpass, se empleó la métrica Laplacian Sharpness Ratio (LSR), que cuantifica el nivel de nitidez en la imagen mediante el análisis de bordes y detalles finos, siendo un indicador clave de la calidad visual en imágenes biomédicas. En este análisis, se compararon dos configuraciones de la máscara de pasa banda: \texttt{low\_scale=5, high\_scale=10} y \texttt{low\_scale=15, high\_scale=20}.

Los resultados indican que la configuración \textit{Máscara de pasa banda con low\_scale=5, high\_scale=10} obtuvo una puntuación LSR de 54.92, mientras que la configuración con \texttt{low\_scale=15, high\_scale=20} alcanzó un valor de 37.54. Dado que un mayor valor de LSR refleja una mayor nitidez y, por ende, mejor calidad visual, se concluye que la primera configuración ofrece un desempeño superior según esta métrica.

Es importante señalar que, aunque se evaluaron otras métricas de calidad, no se encontró suficiente evidencia estadística para afirmar que existan diferencias significativas entre ambas configuraciones en dichas métricas. Por consiguiente, la selección de la configuración óptima se fundamenta en los resultados obtenidos con LSR, en línea con los criterios objetivos de evaluación adoptados en este estudio.

\subsection{Máscara de escala exponencial}

En el análisis de la máscara de escala exponencial, se compararon dos configuraciones experimentales con valores de \texttt{decay\_rate} de 0.1 y 0.01. Los resultados muestran que ambas configuraciones son equivalentes en términos de las métricas de calidad evaluadas, ya que no se encontró suficiente evidencia estadística para afirmar que sus resultados sean diferentes.

Dado que la configuración con \texttt{decay\_rate} igual a 0.01 presentó un mejor desempeño en tiempos de ejecución, se selecciona esta opción como la más adecuada para aplicaciones donde la eficiencia computacional es prioritaria.

\subsection{Máscara de escala gaussiana}

Se evaluaron tres configuraciones de la máscara de escala gaussiana, variando el parámetro \texttt{sigma} en valores de 1.0, 4.0 y 6.0, manteniendo \texttt{center\_scale=N/2}. Los resultados muestran que las configuraciones con \texttt{sigma=4.0} y \texttt{sigma=6.0} son equivalentes en términos de las métricas de calidad evaluadas, sin diferencias estadísticamente significativas.

Sin embargo, la configuración con \texttt{sigma=1.0} superó a las demás en la mayoría de las métricas relevantes, incluyendo CII, LSR, PSNR, SSIM, LPIPS y RNS, lo que indica una mejor calidad de imagen y menor distorsión. Por tanto, se selecciona la máscara de escala gaussiana con \texttt{center\_scale=N/2, sigma=1.0} como la opción óptima para este grupo experimental, en concordancia con los criterios objetivos de evaluación empleados en este trabajo.

\begin{table}[H]
    \centering
    \caption{Comparativa de métricas para las configuraciones de la máscara de escala gaussiana (\texttt{center\_scale=N/2})}
    \begin{tabular}{|l|c|c|c|}
    \hline
    \textbf{Métrica} & \textbf{sigma=1.0} & \textbf{sigma=4.0} & \textbf{sigma=6.0} \\
    \hline
    CII & 1.4690 & 1.2515 & 1.2515 \\
    LSR & 67.1471 & 19.2774 & 19.2774 \\
    PSNR & 65.7315 & 63.7677 & 63.7677 \\
    SSIM & 0.9976 & 0.9963 & 0.9963 \\
    LPIPS & 0.00021 & 0.00036 & 0.00036 \\
    RNS & 0.1129 & 0.1405 & 0.1405 \\
    BRISQUE & 111.96 & 106.81 & 106.81 \\
    \hline
    \end{tabular}
    \label{tab:gaussian_comparativa}
\end{table}

\section{Selección del mejor modelo}

El proceso de selección del mejor modelo se realizó a partir de una rigurosa comparación entre los experimentos que obtuvieron los mejores resultados en cada una de las categorías evaluadas: umbralizado, máscaras exponenciales y gaussianas, filtros pasa alto y pasa banda, así como la mejora de energía. Para garantizar una evaluación objetiva y exhaustiva, se emplearon múltiples métricas de calidad de imagen, incluyendo LSR (Laplacian Sharpness Ratio), BRISQUE, PSNR, SSIM, LPIPS y RNS, cubriendo aspectos de nitidez, distorsión, artefactos y similitud estructural.

En la primera etapa, se identificaron los experimentos con mejor desempeño dentro de cada grupo, seleccionando los siguientes: \textit{Threshold Coefficients at 0.001}, \textit{Exponential Scale Mask with decay\_rate=0.01}, \textit{Highpass Mask with cutoff\_scale=10}, \textit{Gaussian Scale Mask with center\_scale=N/2, sigma=1.0}, \textit{Bandpass Mask with low\_scale=5, high\_scale=10} y \textit{Enhance Energy by 10}.

Posteriormente, se realizó una comparación cruzada entre estos modelos. Los resultados mostraron que varias configuraciones, como el umbralizado y la mejora de energía, presentaron desempeños equivalentes en la mayoría de las métricas evaluadas, sin diferencias estadísticamente significativas. Sin embargo, a medida que se avanzó en la comparación directa con los otros métodos, emergieron diferencias claras en métricas clave.

La máscara gaussiana con \texttt{center\_scale=N/2, sigma=1.0} destacó consistentemente sobre las demás configuraciones. Superó a los otros modelos en métricas como LSR (67.15 frente a valores significativamente menores en otros métodos), PSNR (65.73), SSIM (0.9976), LPIPS (0.00021), y RNS (0.1129), lo que indica una mayor nitidez, menor distorsión y mejor preservación de la estructura y los detalles de la imagen. Además, aunque en la métrica BRISQUE no obtuvo el valor más bajo absoluto, su desempeño global en el conjunto de métricas fue superior.

Este proceso de selección evidencia la importancia de considerar múltiples dimensiones de calidad de imagen y no basar la decisión únicamente en una métrica aislada. Así, la \textit{Máscara de escala gaussiana con center\_scale=N/2, sigma=1.0} fue seleccionada como el mejor modelo global, al ofrecer el balance óptimo entre nitidez, preservación de detalles y baja distorsión, validando su aplicabilidad en el procesamiento avanzado de imágenes de CT cerebral.

\begin{table}[H]
\centering
\caption{Comparativa de métricas para los mejores modelos de cada categoría}
\resizebox{\textwidth}{!}{
\begin{tabular}{|l|c|c|c|c|c|c|}
\hline
\textbf{Métrica} & \textbf{Umbral 0.001} & \textbf{Exp. 0.01} & \textbf{Pasa alto 10} & \textbf{Gaussiana 1.0} & \textbf{Pasa banda 5-10} & \textbf{Energía 10} \\
\hline
CII & 1.2515 & 1.2515 & -- & 1.4690 & 1.2515 & 1.2515 \\
LSR & 43.41 & 43.41 & 54.92 & 67.15 & 54.92 & 43.41 \\
PSNR & -- & -- & -- & 65.73 & 63.77 & -- \\
SSIM & 0.9972 & 0.9963 & 0.9968 & 0.9976 & 0.9963 & 0.9963 \\
LPIPS & 0.00026 & 0.00036 & 0.00030 & 0.00021 & 0.00036 & 0.00036 \\
RNS & 0.1235 & 0.1405 & 0.1308 & 0.1129 & 0.1405 & 0.1405 \\
BRISQUE & 110.90 & 106.81 & 108.99 & 111.96 & 106.81 & 106.81 \\
\hline
\end{tabular}
}
\label{tab:comparativa_final}
\end{table}

\section{Evaluación cualitativa por especialista}

Con el objetivo de complementar el análisis cuantitativo y validar la relevancia clínica de los métodos propuestos, se realizó una evaluación cualitativa a través de una encuesta dirigida a una especialista en radiología. Para este fin, se seleccionaron imágenes mejoradas utilizando el mejor método identificado en la comparación cuantitativa (máscara gaussiana) y el método con menor desempeño (umbralizado).

La encuesta consistió en 32 pares de imágenes, cada par compuesto por una imagen mejorada con la máscara gaussiana y otra con el método de umbralizado. Los pares fueron seleccionados aleatoriamente de distintas secciones del conjunto de datos, y el orden de presentación fue completamente aleatorio para evitar cualquier sesgo y garantizar que la especialista no pudiera identificar el método aplicado en cada caso.

Los resultados obtenidos revelaron que, en 19 casos, ambos métodos fueron considerados equivalentes por la evaluadora. En 9 casos, el método de umbralizado fue valorado como “mucho mejor” y, en 3 casos, la máscara gaussiana recibió dicha calificación.

Para analizar estadísticamente estas preferencias, se aplicó la prueba de McNemar, adecuada para comparar proporciones en datos pareados y determinar si existen diferencias significativas en la preferencia entre dos métodos\cite{mcnemar}. El valor del estadístico de McNemar fue 2.0833, con un valor-p de 0.1489. Según estos resultados, aunque se observó una preferencia por el método de umbralizado en los casos discordantes (50.0\% de diferencia), esta diferencia no alcanzó significancia estadística al nivel convencional ($p > 0.05$).

Estos hallazgos sugieren que, desde la perspectiva cualitativa de la especialista, no existe una diferencia significativa en la percepción de calidad entre ambos métodos en el contexto evaluado.