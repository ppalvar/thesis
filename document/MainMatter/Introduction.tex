\chapter*{Introducción}\label{chapter:introduction}
En la actualidad, las herramientas diagnósticas han alcanzado un notable desarrollo tecnológico, lo que permite una detección más precisa de diversas enfermedades, sin embargo, muchos de los procedimientos utilizados son altamente invasivos para los pacientes lo que compromete su salud en muchos casos, y en ocasiones no es suficientemente efectivo para poder realizar el diagnóstico. En este contexto, la tomografía computarizada (CT, por sus siglas en inglés), es una poderosa herramienta para la visualización indirecta de los tejidos y órganos internos de los pacientes, sin necesidad de una observación directa mediante acceso quirúrgico. Sin embargo, cuando se trata de observar tejidos blandos y lesiones pequeñas (e.g., tumores o hemorragias de varios milímetros en zonas del cerebro) suele ser necesario introducir al organismo un agente contrastante radiactivo como una solución de yodo \cite{InsideRadiologyICCM}. Este método es efectivo en muchos casos ya que proporciona una imagen con mejoras en el contraste, principalmente en zonas con múltiples vasos sanguíneos. Por otro lado, el método puede tener complicaciones para la salud de los pacientes con enfermedades tiroideas o, en casos extremos, provocar ataques cardíacos \cite{IodineTyroids,IodineHeathAttack}.

Dado que las tomografías en muchos casos sí son capaces de captar los detalles necesarios para realizar un diagnóstico, pero estos no son apreciables debido a problemas de contraste, ruido o emborronamiento (blur), resulta necesario la mejora de estas imágenes mediante métodos computacionales precisos, en particular, imágenes de CT del cerebro, que al ser un órgano gelatinoso donde las lesiones suelen ser pequeñas. 

En la literatura se reportan métodos de mejora de imágenes que utilizan redes neuronales en alguna de sus variantes. Este es el caso también de las imágenes de CT, en las cuales se logran resultados muy satisfactorios en ese campo. Los estudios más recientes (2022-2025) demuestran que las técnicas basadas en inteligencia artificial (AI, por sus siglas en inglés) mejoran significativamente la calidad de imagen y mantienen una alta precisión diagnóstica, incluso en estudios con dosis reducidas de radiación \cite{AISurveyOnImageQuality}. Aunque los algoritmos de AI han demostrado potencial en el diagnóstico médico, su implementación efectiva requiere grandes volúmenes de datos anotados por especialistas, así como recursos computacionales avanzados. En contextos con limitaciones de infraestructura —como el sistema de salud cubano—, estas condiciones rara vez se cumplen, especialmente en el caso de bases de datos imagenológicas de CT. Por ello, resulta crítico explorar métodos numéricos tradicionales o novedosos, los cuales, a pesar de ofrecer resultados inferiores en algunos escenarios , presentan ventajas clave: menor dependencia de datos, menor costo computacional y posibilidad de despliegue en dispositivos móviles.

En este contexto, uno de los métodos numéricos de procesamiento de imágenes más actuales es la transformada curvelet (2DCT, por sus siglas en inglés), que ofrece mayor precisión para analizar imágenes con respecto a la transformada wavelet tradicional, pues puede detectar bordes en ángulos y distintas escalas \cite{FastCurveletTransform}. Sin embargo, al considerar las escalas y ángulos como una generalización de tiempo-frecuencia usado en el análisis wavelet, la resolución temporal obtenida es limitada y ofrece grandes áreas de incertidumbre que dificultan la creación de algoritmos que utilicen directamente la 2DCT.

Con el objetivo de mejorar esta resolución (escalar y angular), en esta tesis se utilizará la transformada de \emph{synchrosqueezed} (SST, por sus siglas en inglés), que es un método de reasignación de frecuencias que permite refinar la representación obtenida mediante 2DCT \cite{SynchrosqueezedCurveletTransform}. Resulta aún de mayor interés este método debido a la escasa investigación realizada al respecto en años recientes, en particular, para procesar imágenes de CT.

Como objetivo de este trabajo de tesis, se diseñaron varios experimentos sobre la descomposición SST de un conjunto de imágenes con el objetivo de obtener mejoras al calcular la función inversa (ISST). Para ello, se utilizó la implementación de SST e ISST existentes en \texttt{SynLab} \cite{SynchrosqueezedCurveletTransform,SynchrosqueezedCurveletTransform_SynLab}, adaptado a \texttt{Python 3.13} mediante la librería \texttt{oct2py} \cite{oct2py}. Los experimentos a realizar son descritos en la Tabla~\ref{tab:experimentos}, estos serán explicados con más detalle en las secciones correspondientes.

\begin{table}[h]
    \centering
    \caption{Configuraciones experimentales y sus parámetros.}
    \label{tab:experimentos}
    \begin{tabular}{>{\raggedright}p{4cm}p{6cm}}
    \toprule
    \textbf{Tipo de Experimento} & \textbf{Parámetros} \\ 
    \midrule
    Umbralizado & threshold = 0.1, 0.05, 0.001 \\
    \midrule
    Mejora de energía & enhancement\_factor = 3.0, 1.5, 10 \\
    \midrule
    Filtro pasa-alto & cutoff\_scale = 5, 10 \\
    \midrule
    Filtro pasa-banda & low\_scale = 5 (high\_scale = 10), 15 (20) \\
    \midrule
    Máscara gaussiana & center\_scale = N/2, sigma = 1.0, 4.0, 6.0 \\
    \midrule
    Máscara exponencial & decay\_rate = 0.1, 0.01 \\
    \bottomrule
    \end{tabular}
    \footnotesize{\\*N/2: mitad de las escalas disponibles en la descomposición.}
\end{table}

Para evaluar los resultados de los experimentos se utilizaron un conjunto de métricas (Tabla~\ref{tab:metricas}) agrupadas en tres categorías:
\begin{enumerate}
    \item \textbf{Métricas de mejora}: miden el incremento en la calidad de la imagen.
    \item \textbf{Métricas de distorsión}: mide la distorsión de la imagen con respecto a la original, tanto en sus características semánticas como en otras como el ruido introducido.
    \item \textbf{Métricas de artefactos}: miden cuántos artefactos tiene la imagen reconstruida con respecto a la original.
\end{enumerate}

\begin{table}[h]
    \centering
    \caption{Métricas de evaluación agrupadas por categorías\cite{Metrics}}
    \label{tab:metricas}
    \begin{tabular}{p{3cm}p{8cm}p{3cm}}
    \toprule
    \textbf{Categoría} & \textbf{Métrica} & \textbf{Descripción} \\ 
    \midrule
    \multirow{3}{*}{Mejora} 
    & CII (\textit{Contrast Improvement Index}) & Índice de mejora de contraste \\
    & LSR (\textit{Laplacian Sharpness Ratio}) & Cociente de nitidez Laplaciana \\
    & FSIM (\textit{Feature Similarity Index}) & Similaridad de características \\
    \midrule

    \multirow{3}{*}{Distorsión}
    & PSNR (\textit{Peak Signal-to-Noise Ratio}) & Relación señal-ruido \\
    & SSIM (\textit{Structural Similarity Index}) & Similaridad estructural \\
    & LPIPS (\textit{Learned Perceptual Image Patch Similarity}) & Similaridad perceptual \\
    \midrule

    \multirow{2}{*}{Artefactos}
    & RNS (Residual Noise Standard Deviation) & Desviación estándar del ruido residual \\
    & BRISQUE (Blind/Referenceless Image Spatial Quality Evaluator \cite{BRISQUE}) & Evaluador ciego de calidad \\
    \bottomrule
    \end{tabular}
\end{table}

El presente trabajo tiene como objetivo principal desarrollar y evaluar un método numérico basado en SST aplicada a curvelets (2DCT) para mejorar la calidad de imágenes de tomografía computarizada del cerebro, con especial énfasis en la visualización de tejidos blandos y lesiones pequeñas sin necesidad de agentes de contraste.

A través de los experimentos diseñados (Tabla~\ref{tab:experimentos}) y utilizando métricas cuantitativas validadas (Tabla~\ref{tab:metricas}), se busca: (1) optimizar la reconstrucción inversa (ISST) para preservar características clínicamente relevantes, (2) reducir artefactos y distorsiones en las imágenes procesadas, y (3) mejorar el contraste, nitidez y calidad general de la imagen para los estándares de un profesional del área.

Los resultados obtenidos podrían ofrecer una alternativa no invasiva para mejorar el diagnóstico por imágenes, particularmente en pacientes con contraindicaciones para el uso de contrastes yodados, lo que contribuye así al avance de técnicas computacionales en radiología médica.

En el contexto actual de rápida adopción de la AI en diagnóstico médico, este trabajo aporta dos contribuciones clave: primero, establece un marco numérico reproducible que podría servir como capa de preprocesamiento para modelos de aprendizaje profundo, lo que pudiera potenciar su eficiencia al reducir la complejidad del espacio de características; segundo, provee un enfoque interpretable que mitiga el frecuente ``efecto caja negra'' de los sistemas basados puramente en AI.

Los resultados podrían influir en el desarrollo de sistemas híbridos AI-métodos numéricos, particularmente relevantes para entornos clínicos con limitaciones tecnológicas o donde se priorice la transparencia diagnóstica. Además, la metodología propuesta podría extenderse a otras modalidades de imagen médica más allá de la tomografía computarizada.

Esta tesis está organizada en tres capítulos. El primero introduce los conceptos fundamentales para la comprensión del trabajo, incluyendo bases teóricas de procesamiento de imágenes, aprendizaje automático y humanidades digitales. Además, presenta una revisión del estado del arte, analizando herramientas existentes y seleccionando aquellas relevantes para esta investigación.

El segundo capítulo detalla el desarrollo de la aplicación web, abarcando su arquitectura, metodología de implementación, funcionalidades e integración del modelo de OCR. Se incluyen representaciones visuales de sus interfaces principales.

El tercer capítulo describe la implementación del modelo de OCR, especificando las técnicas de procesamiento de imágenes evaluadas, la estructura del modelo y los resultados experimentales. Adicionalmente, se valida el modelo mediante un conjunto de Actas de la SEAP previamente transcritas, con el fin de estimar su efectividad en aplicaciones futuras. El documento concluye con un análisis de los hallazgos y recomendaciones para investigaciones posteriores.

Esta tesis forma parte del proyecto de investigación ``Wavelets, frames, técnicas espectrales, ecuaciones en derivadas parciales y aprendizaje automático científico en el análisis de imágenes'', asociado al Programa Nacional de Ciencias Básicas, Código PN223LH010-036, Ministerio de Ciencia, Tecnología y Medio Ambiente (CITMA), Cuba, 2024-2026.
\addcontentsline{toc}{chapter}{Introducción}
