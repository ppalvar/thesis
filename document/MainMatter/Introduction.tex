\chapter*{Introducción}\label{chapter:introduction}
En la actualidad, las herramientas diagnósticas han alcanzado un notable desarrollo tecnológico, lo que permite una detección más precisa de diversas enfermedades \cite{semmlow2008biosignal}, sin embargo, muchos de los procedimientos utilizados son altamente invasivos para los pacientes lo que compromete su salud en muchos casos, y en ocasiones no es suficientemente efectivo para poder realizar el diagnóstico.

En este contexto, la tomografía computarizada (CT, por sus siglas en inglés) es una poderosa herramienta para la visualización indirecta de los tejidos y órganos internos de los pacientes, sin necesidad de una observación directa mediante acceso quirúrgico. Sin embargo, cuando se trata de observar tejidos blandos y lesiones pequeñas (e.g., tumores o hemorragias de varios milímetros en zonas del cerebro) suele ser necesario introducir al organismo un agente contrastante radiactivo como una solución de yodo \cite{InsideRadiologyICCM}.

Este método es efectivo en muchos casos, ya que proporciona una imagen con mejoras en el contraste, principalmente en zonas con múltiples vasos sanguíneos. Por otro lado, el método puede tener complicaciones para la salud de los pacientes con enfermedades tiroideas o, en casos extremos, provocar ataques cardíacos \cite{IodineTyroids,IodineHeathAttack}.

Dado que las tomografías, en muchos casos, son capaces de captar los detalles necesarios para realizar un diagnóstico, pero estos no son apreciables debido a problemas de contraste, ruido o emborronamiento de la imagen, resulta necesario la mejora digital de estas imágenes mediante métodos computacionales precisos, en particular, imágenes de CT del cerebro, que al ser un órgano gelatinoso donde las lesiones suelen ser pequeñas. 

En la literatura se reportan métodos de mejora de imágenes que utilizan redes neuronales en alguna de sus variantes \cite{ULTRA,DLR,LEARN++,EDCNN}. Este es el caso también de las imágenes de CT, en las cuales se logran resultados muy satisfactorios en ese campo. Los estudios más recientes (2022-2025) demuestran que las técnicas basadas en inteligencia artificial (IA, por sus siglas) mejoran significativamente la calidad de imagen y mantienen una alta precisión diagnóstica, incluso en estudios con dosis reducidas de radiación \cite{AISurveyOnImageQuality}.

Aunque los algoritmos de IA han demostrado potencial en el diagnóstico médico, su implementación efectiva requiere grandes volúmenes de datos anotados por especialistas, así como recursos computacionales avanzados. En contextos con limitaciones de infraestructura —como el sistema de salud cubano—, estas condiciones rara vez se cumplen, especialmente en el caso de bases de datos imagenológicas de CT.

Por ello, resulta crítico explorar métodos numéricos tradicionales y novedosos, los cuales, a pesar de ofrecer resultados inferiores en algunos escenarios , presentan ventajas clave: menor dependencia de datos, menor costo computacional y posibilidad de despliegue en dispositivos móviles.

En este contexto, uno de los métodos numéricos de procesamiento de imágenes más actuales es la transformada \emph{curvelet} (2DCT, por sus siglas en inglés), que ofrece mayor precisión para analizar imágenes con respecto a la transformada \emph{wavelet} tradicional \cite{Flandrin2018}, pues puede detectar bordes en ángulos y distintas escalas \cite{FastCurveletTransform}. Sin embargo, al considerar las escalas y ángulos como una generalización de tiempo-frecuencia usado en el análisis \emph{wavelet}, la resolución temporal obtenida es limitada y ofrece grandes áreas de incertidumbre que dificultan la creación de algoritmos que utilicen directamente la 2DCT.

La transformada de \emph{synchrosqueezed} (SST, por sus siglas en inglés) es un método de reasignación de frecuencias que permite refinar la representación obtenida mediante 2DCT \cite{SynchrosqueezedCurveletTransform}. En este método se obtiene una matriz de coeficientes (también referida como energía SST) que representa la energía de la imagen en cada escala y ángulo.

La hipótesis central de esta tesis es la siguiente: si se realiza una modificación de la energía SST, ¿se puede obtener una mejora en la calidad de la imagen al aplicar la función inversa (ISST) de la transformada?

El presente trabajo de tesis tiene como objetivo principal desarrollar un método numérico basado en SST aplicada a curvelets (2DCT) para mejorar la calidad de imágenes de tomografía computarizada del cerebro, con especial énfasis en la visualización de tejidos blandos y lesiones pequeñas sin necesidad de agentes de contraste.

Como objetivos específicos se plantean:

\begin{enumerate}
    \item Implementar la transformada SST-2DCT e ISST sobre imágenes de CT cerebral con \texttt{Python 3.13}.
    \item Diseñar y ejecutar experimentos enfocados a tener una mejor representación de las características deseadas de la imagen, mediante la modificación de sus matrices de energía.
    \item Evaluar cuantitativamente los resultados mediante métricas de mejora, a través de conjuntos de métricas enfocadas en medir la mejora, la distorsión y el ruido de los resultados.
    \item Comparar la reconstrucción inversa (ISST) optimizada con métodos tradicionales de mejoramiento de imágenes de CT y métodos más modernos.
    \item Analizar la viabilidad del método en entornos con recursos computacionales limitados y proponer recomendaciones de implementación para futuros trabajos.
\end{enumerate}

Con estos objetivos se busca: (1) optimizar la reconstrucción inversa (ISST) para preservar características clínicamente relevantes, (2) reducir artefactos y distorsiones en las imágenes procesadas, y (3) mejorar el contraste, nitidez y calidad general de la imagen para los estándares de un profesional del área.

Para simplificar los pasos de implementación, se utilizó la implementación de SST e ISST existentes en \texttt{SynLab} \cite{SynchrosqueezedCurveletTransform,SynchrosqueezedCurveletTransform_SynLab}, adaptado a \texttt{Python 3.13} mediante la librería \texttt{oct2py} \cite{oct2py}.

Para evaluar los resultados de los experimentos se utilizaron un conjunto de métricas agrupadas en tres categorías:

\begin{itemize}
    \item \textbf{Métricas de mejora}: miden el incremento en la calidad de la imagen.
    \item \textbf{Métricas de distorsión}: mide la distorsión de la imagen con respecto a la original, tanto en sus características semánticas como en otras como el ruido introducido.
    \item \textbf{Métricas de artefactos}: miden cuántos artefactos tiene la imagen reconstruida con respecto a la original.
\end{itemize}

Los resultados obtenidos podrían ofrecer una alternativa no invasiva para mejorar el diagnóstico por imágenes, particularmente en pacientes con contraindicaciones para el uso de contrastes yodados, lo que contribuye al avance de técnicas computacionales en radiología médica.

En el contexto actual de rápida adopción de la IA en el diagnóstico médico, este trabajo aporta dos contribuciones clave: primero, establece un marco numérico reproducible que podría servir como capa de preprocesamiento para modelos de aprendizaje profundo. Esto pudiera potenciar su eficiencia al reducir la complejidad del espacio de características. Segundo, provee un enfoque interpretable que mitiga el frecuente ``efecto caja negra'' de los sistemas basados puramente en IA.

Los resultados podrían influir en el desarrollo de sistemas híbridos IA-métodos numéricos, particularmente relevantes para entornos clínicos con limitaciones tecnológicas o donde se priorice la transparencia diagnóstica. Además, la metodología propuesta podría extenderse a otras modalidades de imagen médica más allá de la tomografía computarizada.

Esta tesis está organizada en tres capítulos. El primero introduce los conceptos fundamentales para la comprensión del trabajo, incluyendo bases teóricas de procesamiento de imágenes, la transformada \emph{curvelet} y la transformada \emph{synchrosqueezed}. Además, presenta una revisión del estado del arte, donde se analizan herramientas existentes y seleccionando aquellas relevantes para esta investigación.

El segundo capítulo presenta en detalle la implementación del método propuesto para la mejora de imágenes. En esta sección se describen los conjuntos de datos empleados, las métricas de evaluación seleccionadas y la justificación de la metodología adoptada. Asimismo, se expone el procedimiento seguido para la integración de las técnicas de procesamiento de imágenes, especificando los parámetros utilizados y la estructura general del modelo desarrollado.

El tercer capítulo expone los resultados estadísticos obtenidos a partir de la experimentación. Se presentan de manera sistemática los resultados derivados de las distintas configuraciones evaluadas. Además, los resultados fueron sometidos a la evaluación de una especialista en el área, con el fin de validar la calidad de las imágenes procesadas desde una perspectiva experta. Finalmente, se realiza un análisis crítico de los hallazgos, resaltando las ventajas y limitaciones observadas, y se ofrecen recomendaciones para futuras líneas de investigación.

Esta tesis forma parte del proyecto de investigación ``Wavelets, frames, técnicas espectrales, ecuaciones en derivadas parciales y aprendizaje automático científico en el análisis de imágenes'', asociado al Programa Nacional de Ciencias Básicas, Código PN223LH010-036, Ministerio de Ciencia, Tecnología y Medio Ambiente (CITMA), Cuba, 2024-2026.

\addcontentsline{toc}{chapter}{Introducción}
